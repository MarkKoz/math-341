\documentclass[letterpaper,12pt]{article}
\usepackage[section=]{mathhw}
\usepackage{physics}
\usepackage{istgame}

\title{MATH 341 - Practice Final}

\begin{document}

\maketitle
\begin{enumerate}
  \item[1.]
    Suppose that the income of a certain country has probability distribution function (PDF):
    \begin{align*}
      f(x) = \begin{cases}
        \frac{k}{x^4} & 10 \le x \\
        0             & \text{otherwise}
      \end{cases}
    \end{align*}
    \begin{enumerate}
      \item[a.]
        Find the value of $k$ which makes $f(x)$ a legitimate pdf.
        \begin{align*}
          1 &= \int_{10}^{\infty} \frac{k}{x^4} \dd{x} \\
          &= k \int_{10}^{\infty} x^{-4} \dd{x} \\
          &= -\frac{k}{3} \cdot x^{-3} \big\rvert_{10}^{\infty} \\
          &= -\frac{k}{3} \left[\lim_{b \to \infty} b^{-3} - 10^{-3}\right] \\
          &= -\frac{k}{3} [0 - 10^{-3}] \\
          &= \frac{k}{3 \cdot 10^3} \\
          3000 &= k
        \end{align*}
      \item[b.]
        Find the mean income and the standard deviation of income.
        \begin{align*}
          E(X) &= \int_{10}^{\infty} x \cdot \frac{3000}{x^4} \dd{x} \\
          &= 3000 \int_{10}^{\infty} x^{-3} \dd{x} \\
          &= -\frac{3000}{2} \cdot x^{-2} \big\rvert_{10}^{\infty} \\
          &= -1500 \left[\lim_{b \to \infty} b^{-2} - 10^{-2}\right] \\
          &= -1500 [0 - 10^{-2}] \\
          &= 1500 \cdot 10^{-2} \\
          &= 15
        \end{align*}
        \begin{align*}
          E(X^2) &= \int_{10}^{\infty} x^2 \cdot \frac{3000}{x^4} \dd{x} \\
          &= 3000 \int_{10}^{\infty} x^{-2} \dd{x} \\
          &= -3000 \cdot x^{-1} \big\rvert_{10}^{\infty} \\
          &= -3000 \left[\lim_{b \to \infty} b^{-1} - 10^{-1}\right] \\
          &= -3000 [0 - 10^{-1}] \\
          &= 3000 \cdot 10^{-1} \\
          &= 300
        \end{align*}
        \begin{align*}
          \sigma &= \sqrt{V(X)} \\
          &= \sqrt{E(X^2) - [E(X)]^2} \\
          &= \sqrt{300 - 15^2} \\
          &= \sqrt{75} \\
          &\approx 8.66
        \end{align*}
      \item[c.]
        Find $P(20 < X \le 40)$.
        \begin{align*}
          P(20 < X \le 40) &= \int_{20}^{40} \frac{3000}{x^4} \dd{x} \\
          &= 3000 \int_{20}^{40} x^{-4} \dd{x} \\
          &= -\frac{3000}{3} \cdot x^{-3} \big\rvert_{20}^{40} \\
          &= -1000 [40^{-3} - 20^{-3}] \\
          &= -\frac{1000}{64000} + \frac{1000}{8000} \\
          &= -\frac{1}{64} + \frac{8}{64} \\
          &= \frac{7}{64} \\
          &= -.109375
        \end{align*}
    \end{enumerate}
  \item[2.]
    A company that manufactures video games produces two models Xbox Series S and Xbox Series X. Over the past month, 40\% of the video games sold have been Series S. Of those buying the Series S, 30\% purchase an extended warranty, whereas 50\% of all Series X purchasers do so.
    \begin{enumerate}
      \item[a.]
        What is a probability that a randomly selected purchaser is Series X and does not have an extended warranty? [Draw the tree diagram].
        \begin{align*}
          P(W^\prime|X) \cdot P(X) = P(W^\prime \cap X) = .30
        \end{align*}
        \begin{istgame}
          \xtdistance{30mm}{30mm}
          \setistgrowdirection'{east}

          \istroot(0)+20mm..40mm+
            \istb{P(S) = .40}[above,sloped]
            \istb{P(X) = .60}[below,sloped]
          \endist

          \istroot(1)(0-1)<180>
            \istb{P(W|S) = .30}[above,sloped]{P(W|S) \cdot P(S) = P(W \cap S) = .12}
            \istb{P(W^\prime|S) = .70}[below,sloped]{P(W^\prime|S) \cdot P(S) = P(W^\prime \cap S) = .28}
          \endist

          \istroot(2)(0-2)<0>
            \istb{P(W|X) = .50}[above,sloped]{P(W|X) \cdot P(X) = P(W \cap X) = .30}
            \istb{P(W^\prime|X) = .50}[below,sloped]{P(W^\prime|X) \cdot P(X) = P(W^\prime \cap X) = .30}
          \endist
        \end{istgame}
      \item[b.]
        What is a probability that a randomly selected purchaser does not have an extended warranty?
        \begin{align*}
          P(W^\prime) &= P(W^\prime \cap X) + P(W^\prime \cap S) \\
          &= .28 + .30 \\
          &= .58
        \end{align*}
      \item[c.]
        If a randomly selected purchaser has an extended warranty, how likely is it that he or she has an Xbox Series S?
        \begin{align*}
          P(S|W) &= \frac{P(S \cap W)}{P(W)} \\
          &= \frac{.12}{1 - P(W^\prime)} \\
          &= \frac{.12}{.42} \\
          &\approx .2857
        \end{align*}
    \end{enumerate}
  \item[3.]
    CSUN Statistics department purchased 24 TI scientific calculators from a dealer in order to have a supply on hand for tests for use by students who forget to bring their own. Although the department was not aware of this, seven of the calculators were defective and gave incorrect answers to calculations. When a test is being written, students who have forgotten their own calculators are allowed to select at random one of the Department's. Suppose at the first test of the term, four students forgot to bring their calculators.
    \begin{enumerate}
      \item[a.]
        What is the probability that at least two of these students select a defective calculator?
        \\
        There are two outcomes: a student chooses a defective calculator, or a student chooses a functional calculator. Because there are only two outcomes, and sampling is done \textit{without} replacement, the hypergeometric distribution $h(x; n, M, N)$ should be used. Let $n = 4$, $M = 7$, and $N = 24$.
        \begin{align*}
          P(X \ge 2) &= \sum_{x = 2}^{4} \frac{\binom{M}{x} \binom{N - M}{n - x}}{\binom{N}{n}} \\
          &= \sum_{x = 2}^{4} \frac{\binom{7}{x} \binom{24 - 7}{4 - x}}{\binom{24}{4}} \\
          &= \frac{\binom{7}{2} \binom{17}{2}}{\binom{24}{4}} + \frac{\binom{7}{3} \binom{17}{1}}{\binom{24}{4}} + \frac{\binom{7}{4} \binom{17}{0}}{\binom{24}{4}} \\
          &= \frac{(21 \cdot 136) + (35 \cdot 17) + (35 \cdot 1)}{10626} \\
          &= \frac{2856 + 595 + 35}{10626} \\
          &= \frac{3486}{10626} \\
          &\approx .3281
        \end{align*}
      \item[b.]
        What is the probability that at most one of these students select a defective calculator?
        \begin{align*}
          P(X \le 1) &= 1 - P(X \ge 2) \\
          &\approx 1 - .3281 \\
          &\approx .6719
        \end{align*}
      \item[c.]
        What is the pmf of defective calculator?
        \begin{align*}
          h(x; 4, 7, 24) &= \frac{\binom{7}{x} \binom{17}{4 - x}}{10626}
        \end{align*}
    \end{enumerate}
  \item[4.]
    If a symmetric die is tossed 36 times, by using normal approximation find the probability that
    \begin{enumerate}
      \item[a.]
        it comes up number 3 more than 5 times;
        \begin{align*}
          P(X > 5) &= 1 - P(X \le 5) \\
          &\approx 1 - \Phi\left(\frac{5 + .5 - np}{\sqrt{npq}}\right) \\
          &\approx 1 - \Phi\left(\frac{5.5 - \frac{36}{6}}{\sqrt{\frac{36}{6} \cdot \frac{5}{6}}}\right) \\
          &\approx 1 - \Phi\left(\frac{5.5 - 6}{\sqrt{5}}\right) \\
          &\approx 1 - \Phi(-.22) \\
          &\approx 1 - .4129 \\
          &\approx .5871
        \end{align*}
      \item[b.]
        the number of 3s is between 5 and 10 (inclusive);
        \begin{align*}
          P(5 \le X \le 10) &\approx \Phi\left(\frac{10 + .5 - 6}{\sqrt{5}}\right) - \Phi\left(\frac{5 + .5 - 6}{\sqrt{5}}\right) \\
          &\approx \Phi(2.01) - \Phi(-.22) \\
          &\approx .9978 - .4129 \\
          &\approx .5849
        \end{align*}
      \item[c.]
        the number of 6s is exactly 5.
        \begin{align*}
          P(X = 5) &\approx \Phi\left(\frac{5 + .5 - 6}{\sqrt{5}}\right) - \Phi\left(\frac{5 - .5 - 6}{\sqrt{5}}\right) \\
          &\approx \Phi(-.22)  - \Phi(-.67) \\
          &\approx .4129 - .2514 \\
          &\approx .1615
        \end{align*}
        Note the probability ($\frac{1}{6}$) for rolling a 6 is the same as for a 3 or any other number since the die is symmetric.
    \end{enumerate}
  \item[5.]
    The programmer tests the code until three successive results. Suppose that the success probability is 0.7.
    \begin{enumerate}
      \item[a.]
        What is the probability that he or she will test the code four times?
        \\
        The probability of $n = 4$ trials given $r = 3$ successes can be determined by using the negative binomial distribution with $x = n - r = 1$ failure.
        \begin{align*}
          P(X = 1) &= nb(x; r, p) \\
          &= \binom{x + r - 1}{x} p^r (1 - p)^x \\
          &= \binom{1 + 3 - 1}{1} .7^3 (1 - .7)^1 \\
          &= \binom{3}{1} \times .343 \times .3^1 \\
          &= 3 \times .343 \times .3 \\
          &= .3087
        \end{align*}
      \item[b.]
        What are the mean and SD of number of tests?
        \\
        Because the number of tests is discrete, the ceiling function is used on the mean.
        \begin{align*}
          E(X) &= \bigg\lceil \frac{r}{p} \bigg\rceil = \bigg\lceil \frac{3}{.7} \bigg\rceil = 5
        \end{align*}
        \begin{align*}
          \sigma &= \sqrt{V(X)} \\
          &= \sqrt{\frac{r(1 - p)}{p^2}} \\
          &= \sqrt{\frac{3(1 - .7)}{.7^2}} \\
          &= \frac{\sqrt{.9}}{.7} \\
          &\approx 1.36
        \end{align*}
      \item[c.]
        What is the probability that the number of tests is at most 4?
        \\
        Note that the minimum number of trials for 3 successes is 3 trials. Again, $x = n - r$ is used.
        \begin{align*}
          P(X \le 1) &= P(X = 0) + P(X = 1) \\
          &= \binom{0 + 3 - 1}{0} .7^3 (1 - .7)^0 + .3087 \\
          &= (1 \times .343 \times 1) + .3087 \\
          &= .343 + .3087 \\
          &= .6517
        \end{align*}
    \end{enumerate}
  \item[6.]
    In a sample of 123 hip surgeries of a certain type, the average surgery time was 136.9 minutes with a standard deviation of 22.6 minutes.
    \begin{enumerate}
      \item[a.]
        Find a 98\% confidence interval for the mean time.
        \begin{align*}
          \bar{x} \pm z_{\alpha / 2} \cdot \frac{s}{\sqrt{n}} &= 136.9 \pm z_{.01} \cdot \frac{22.6}{\sqrt{123}} \\
          &\approx 136.9 \pm 2.33 \cdot 2.04 \\
          &\approx 136.9 \pm 4.75 \\
          &\approx (132.15, 141.65)
        \end{align*}
      \item[b.]
        Determine the minimum required sample size if you want to be 95\% confident that the sample mean is within 2.5 minutes of the population mean.
        \begin{align*}
          n &= \bigg\lceil \left(2z_{\alpha / 2} \cdot \frac{s}{w}\right)^2 \bigg\rceil \\
          &= \bigg\lceil \left(z_{.025} \cdot \frac{22.6}{2.5}\right)^2 \bigg\rceil \\
          &\approx \big\lceil (1.96 \cdot 9.04)^2 \big\rceil \\
          &\approx 314
        \end{align*}
    \end{enumerate}
  \item[7.]
    If $A$ and $B$ are independent events, show that so are events $A^c$ and $B^c$.
  \item[8.]
    A certain chemical pollutant in the Genesee River has been constant for several years with mean 34 ppm (parts per million) and standard deviation $\sigma = 8$ ppm. A group of factory representatives whose companies discharge liquids into the river is now claiming that they have lowered the average with improved filtration devices. A group of environmentalists will test to see if this is true at the 0.01 level of significance. Assume that their sample of size 50 gives a mean of 32.5 ppm. \\
    Perform a hypothesis test and state your decision.
  \item[9.]
    Researchers studying the effects of diet on growth would like to know if a vegetarian diet affects the height of a child. The average height for all six-year old children is 42.5 inches. The researchers randomly selected 12 vegetarian children that are six years old. The average height of those six-year old vegetarian children is 45.75 inches with a standard deviation of 3.8 inches. Conduct a hypothesis test to determine whether there is overwhelming evidence at $\alpha = .05$ that six-year old vegetarian children are not the same height as other six-year old children. Assume that the height of six-year old children is approximately normally distributed.
  \item[10.]
    The Online Exam from Applied Statistics consists of 6 questions. Statistics show that there is a 75\% chance that the student will answer to any one of Exam problems correctly. If the number of attempts for each question is unlimited, find the following probabilities
    \begin{enumerate}
      \item[a.]
        the student will correctly answer the first question after the 4th attempt;
      \item[b.]
        the student will correctly answer three questions after 10 total attempts.
      \item[c.]
        What are the average number and SD of attempts up to when the student answers all the questions correctly?
    \end{enumerate}
\end{enumerate}

\end{document}
