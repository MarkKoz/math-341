\documentclass[letterpaper,12pt]{article}
\usepackage[section=2.2]{mathhw}
\usepackage{braket}

\begin{document}

\maketitle

\begin{enumerate}
  \item[12.]
    Consider randomly selecting a student at a large university, and let $A$ be the event that the selected student has a Visa card and $B$ be the analogous event for MasterCard. Suppose that $P(A) = .6$ and $P(B) = .4$.
    \begin{enumerate}
      \item[a.]
        Could it be the case that $P(A \cap B) = .5$? Why or why not? [\textit{Hint:} See Exercise 24.]
        \\ \\
        No. $P(B) \geq P(A \cap B)$ since $(A \cap B) \subseteq B$. However, $P(B) = .4 \not\geq .5 = P(A \cap B)$. In other words, if the probability of having both is $.5$, then the probability of having a MasterCard must be at least $.5$.
      \item[b.]
        From now on, suppose that $P(A \cap B) = .3$. What is the probability that the selected student has at least one of these two types of cards?
        \begin{align*}
          P(A \cup B) = P(A) + P(B) - P(A \cap B) = .6 + .4 - .3 = .7
        \end{align*}
      \item[c.]
        What is the probability that the selected student has neither type of card?
        \begin{align*}
          P((A \cup B)^\prime) = 1 - P(A \cup B) = 1 - .7 = .3
        \end{align*}
      \item[d.]
        Describe, in terms of $A$ and $B$, the event that the selected student has a Visa card but not a MasterCard, and then calculate the probability of this event.
        \begin{align*}
          P(A \cap B^\prime) = P(A \cup B) - P(B) = .7 - .4 = .3
        \end{align*}
      \item[e.]
        Calculate the probability that the selected student has exactly one of the two types of cards.
        \begin{align*}
          P((A^\prime \cap B) \cup (A \cap B^\prime)) &= P(A^\prime \cap B) + P(A \cap B^\prime) \\
          &= P(A \cup B) - P(A) + P(A \cup B) - P(B) \\
          &= .7 - .6 + .7 - .3 \\
          &= .5
        \end{align*}
    \end{enumerate}
  \item[14.]
    Suppose that 55\% of all adults regularly consume coffee, 45\% regularly consume carbonated soda, and 70\% regularly consume at least one of these two products.
    \\ \\
    Let $P(A) = .55$ for coffee, $P(B) = .45$ for carbonated soda, and $P(A \cup B) = .7$ for at least one of the two.
    \begin{enumerate}
      \item[a.]
        What is the probability that a randomly selected adult regularly consumes both coffee and soda?
        \begin{align*}
          P(A \cup B) &= P(A) + P(B) - P(A \cap B) \\
          .7 &= .55 + .45 - P(A \cap B) \\
          P(A \cap B) &= .55 + .45 - .7 = .3
        \end{align*}
      \item[b.]
        What is the probability that a randomly selected adult doesn’t regularly consume at least one of these two products?
        \begin{align*}
          P((A \cup B)^\prime) = 1 - P(A \cup B) = 1 - .7 = .3
        \end{align*}
    \end{enumerate}
  \item[16.]
  An individual is presented with three different glasses of cola, labeled $C$, $D$, and $P$. He is asked to taste all three and then list them in order of preference. Suppose the same cola has actually been put into all three glasses.
  \begin{enumerate}
      \item[a.]
        What are the simple events in this ranking experiment, and what probability would you assign to each one?
        \begin{align*}
          \mathcal{S} = \Set{CDP, CPD, DCP, DPC, PCD, PDC}
        \end{align*}
        Because the same cola is in all glasses, all six outcomes are equally likely. Hence, they all have the same probability of $\frac{1}{6}$.
      \item[b.]
        What is the probability that $C$ is ranked first?
        \begin{align*}
          P(\Set{CDP, CPD}) = \frac{2}{6} = \frac{1}{3}
        \end{align*}
        Out of the six total outcomes in $\mathcal{S}$, two of them start with $C$.
      \item[c.]
        What is the probability that $C$ is ranked first and $D$ is ranked last?
        \begin{align*}
          P(\Set{CPD}) = \frac{1}{6}
        \end{align*}
    \end{enumerate}
  \item[17.]
  Let $A$ denote the event that the next request for assistance from a statistical software consultant relates to the SPSS package, and let $B$ be the event that the next request is for help with SAS. Suppose that $P(A) = .30$ and $P(B) = .50$.
  \begin{enumerate}
      \item[a.]
        Why is it not the case that $P(A) + P(B) = 1$?
      \item[b.]
        Calculate $P(A^\prime)$.
      \item[c.]
        Calculate $P(A\cup B)$.
      \item[d.]
        Calculate $P(A^\prime \cap B^\prime)$.
    \end{enumerate}
  \item[22.]
    The route used by a certain motorist in commuting to work contains two intersections with traffic signals. The probability that he must stop at the first signal is $.4$, the analogous probability for the second signal is $.5$, and the probability that he must stop at at least one of the two signals is $.7$. What is the probability that he must stop
  \begin{enumerate}
      \item[a.]
        At both signals?
      \item[b.]
        At the first signal but not at the second one?
      \item[c.]
        At exactly one signal?
    \end{enumerate}
  \item[24.]
    Show that if one event $A$ is contained in another event $B$ (i.e., $A$ is a subset of $B$), then $P(A) <= P(B)$. [\textit{Hint:} For such $A$ and $B$, $A$ and $B \cap A^\prime$ are disjoint and $B = A \cup (B \cap A^\prime)$, as can be seen from a Venn diagram.] For general $A$ and $B$, what does this imply about the relationship among $P(A \cap B)$, $P(A)$ and $P(A \cup B)$?
\end{enumerate}

\end{document}
