\documentclass[letterpaper,12pt]{article}
\usepackage[section=4.1]{mathhw}
\usepackage{plotting}

\begin{document}

\maketitle

\begin{enumerate}
  \item[1.]
    The current in a certain circuit as measured by an ammeter is a continuous random variable $X$ with the following density function:
    \begin{align*}
      f(x) = \begin{cases}
        .075x + .2 & 3 \le x \le 5 \\
        0          & \text{otherwise}
      \end{cases}
    \end{align*}
    \begin{enumerate}
      \item[a.]
        Graph the pdf and verify that the total area under the density curve is indeed 1.
        \begin{center}
          \begin{tikzpicture}
            \begin{axis}[
              base,
              width = .5\linewidth,
              enlargelimits = upper,
              xmin = 3,
              xmax = 5,
              ymin = 0.4,
            ]
              \addplot[blue] {(.075 * x) + .2};
            \end{axis}
          \end{tikzpicture}
        \end{center}
        \begin{align*}
           \int_3^5 (.075x + .2)\,dx &= .075\int_3^5 x\,dx + .2\int_3^5 1\,dx \\
           &= .075\frac{x^2}{2}\bigg\rvert_3^5 + .2x\big\rvert_3^5 \\
           &= .075\left(\frac{25}{2} - \frac{9}{2}\right) + .2(5 - 3) \\
           &= 1
        \end{align*}
      \item[b.]
        Calculate $P(X \le 4)$. How does this probability compare to $P(X < 4)$?
        \begin{align*}
          P(X \le 4) &= \int_3^4 (.075x + .2)\,dx \\
          &= .075\int_3^4 x\,dx + .2\int_3^4 1\,dx \\
          &= .075\frac{x^2}{2}\bigg\rvert_3^4 + .2x\big\rvert_3^4 \\
          &= .075\left(\frac{16}{2} - \frac{9}{2}\right) + .2(4 - 3) \\
          &= .4625
        \end{align*}
        $P(X \le 4) = P(X < 4)$ since $P(X = c) = 0$ when $X$ is continuous.
      \item[c.]
        Calculate $P(3.5 \le X \le 4.5)$ and also $P(4.5 < X)$.
        \begin{align*}
          P(3.5 \le X \le 4.5) &= \int_{3.5}^{4.5} (.075x + .2)\,dx \\
          &= .075\int_{3.5}^{4.5} x\,dx + .2\int_{3.5}^{4.5} 1\,dx \\
          &= .075\frac{x^2}{2}\bigg\rvert_{3.5}^{4.5} + .2x\big\rvert_{3.5}^{4.5} \\
          &= .075\left(\frac{20.25}{2} - \frac{12.25}{2}\right) + .2(4.5 - 3.5) \\
          &= .5
        \end{align*}
    \end{enumerate}
  \item[2.]
    Suppose the reaction temperature $X$ (in $^\circ$C) in a certain chemical process has a uniform distribution with $A = -5$ and $B = 5$.
    \begin{enumerate}
      \item[a.]
        Compute $P(X < 0)$.
        \begin{align*}
          P(X < 0) &= \int_{A}^{0} \frac{1}{B - A}\,dx \\
          &= \int_{-5}^{0} \frac{1}{10}\,dx \\
          &= \frac{x}{10}\bigg\rvert_{-5}^{0} \\
          &= \frac{0}{10} - \frac{-5}{10} \\
          &= .5
        \end{align*}
      \item[b.]
        Compute $P(-2.5 < X < 2.5)$.
        \begin{align*}
          P(-2.5 < X < 2.5) &= \int_{-2.5}^{2.5} \frac{1}{B - A}\,dx \\
          &= \int_{-2.5}^{2.5} \frac{1}{10}\,dx \\
          &= \frac{x}{10}\bigg\rvert_{-2.5}^{2.5} \\
          &= \frac{2.5}{10} - \frac{-2.5}{10} \\
          &= .5
        \end{align*}
      \item[c.]
        Compute $P(-2 \le X \le 3)$.
        \begin{align*}
          P(-2 < X < 3) &= \int_{-2}^{3} \frac{1}{B - A}\,dx \\
          &= \int_{-2}^{3} \frac{1}{10}\,dx \\
          &= \frac{x}{10}\bigg\rvert_{-2}^{3} \\
          &= \frac{3}{10} - \frac{-2}{10} \\
          &= .5
        \end{align*}
      \item[d.]
        For $k$ satisfying $-5 < k < k + 4 < 5$, compute $P(k < X < k + 4)$.
        \begin{align*}
          P(k < X < k + 4) &= \int_{k}^{k + 4} \frac{1}{B - A}\,dx \\
          &= \int_{k}^{k + 4} \frac{1}{10}\,dx \\
          &= \frac{x}{10}\bigg\rvert_{k}^{k + 4} \\
          &= \frac{k + 4}{10} - \frac{k}{10} \\
          &= .4
        \end{align*}
    \end{enumerate}
  \item[3.]
    The error involved in making a certain measurement is a continuous rv $X$ with pdf
    \begin{align*}
      f(x) = \begin{cases}
        .09375(4 - x^2) & -2 \le x \le 2 \\
        0               & \text{otherwise}
      \end{cases}
    \end{align*}
    \begin{enumerate}
      \item[a.]
        Sketch the graph of $f(x)$.
        \begin{center}
          \begin{tikzpicture}
            \begin{axis}[
              base,
              width = .5\linewidth,
              enlargelimits = upper,
              xmin = -2,
              xmax = 2,
              ymin = 0,
            ]
              \addplot[blue, smooth] {.09375 * (4 - x^2)};
            \end{axis}
          \end{tikzpicture}
        \end{center}
      \item[b.]
        Compute $P(X > 0)$.
        \begin{align*}
          P(X > 0) &= \int_{0}^{\infty} f(x)\,dt \\
          &= \int_{0}^{2} .09375(4 - x^2)\,dt \\
          &= .09375\left[4\int_{0}^{2} dt - \int_{0}^{2} x^2\,dt\right] \\
          &= .09375\left[4x\big\rvert_{0}^{2} - \frac{x^3}{3}\bigg\rvert_{0}^{2}\right]  \\
          &= .09375\left[4(2 - 0) - \frac{1}{3}(8 - 0)\right] \\
          &= .09375\left[4(2) - \frac{8}{3}\right] \\
          &= .5
        \end{align*}
      \item[c.]
        Compute $P(-1 < X < 1)$.
        \begin{align*}
          P(-1 < X < 1) &= \int_{-1}^{1} f(x)\,dt \\
          &= \int_{-1}^{1} .09375(4 - x^2)\,dt \\
          &= .09375\left[4\int_{-1}^{1} dt - \int_{-1}^{1} x^2\,dt\right] \\
          &= .09375\left[4x\big\rvert_{-1}^{1} - \frac{x^3}{3}\bigg\rvert_{-1}^{1}\right]  \\
          &= .09375\left[4(1 - -1) - \frac{1}{3}(1 - -1)\right] \\
          &= .09375\left[4(2) - \frac{2}{3}\right] \\
          &= .6875
        \end{align*}
      \item[d.]
        Compute $P(X < -.5\ \text{or}\ X > .5)$.
        \begin{align*}
          &P(X < -.5\ \text{or}\ X > .5) \\
          &= 1 - P(-.5 \le X \le .5) \\
          &= 1 - \int_{-.5}^{5} f(x)\,dt \\
          &= 1 - \int_{-.5}^{.5} .09375(4 - x^2)\,dt \\
          &= 1 - .09375\left[4\int_{-.5}^{.5} dt - \int_{-.5}^{.5} x^2\,dt\right] \\
          &= 1 - .09375\left[4x\big\rvert_{-.5}^{.5} - \frac{x^3}{3}\bigg\rvert_{-.5}^{.5}\right]  \\
          &= 1 - .09375\left[4(.5 - -.5) - \frac{1}{3}(.125 - -.125)\right] \\
          &= 1 - .09375\left[4 - \frac{.25}{3}\right] \\
          &= .6328125
        \end{align*}
    \end{enumerate}
  \item[5.]
    A college professor never finishes his lecture before the end of the hour and always finishes his lectures within 2 min after the hour. Let $X =$ the time that elapses between the end of the hour and the end of the lecture and suppose the pdf of $X$ is
    \begin{align*}
      f(x) = \begin{cases}
        kx^2 & 0 \le x \le 2 \\
        0    & \text{otherwise}
      \end{cases}
    \end{align*}
    \begin{enumerate}
      \item[a.]
        Find the value of $k$ and draw the corresponding density curve. [\textit{Hint:} Total area under the graph of $f(x)$ is 1.]
      \item[b.]
        What is the probability that the lecture ends within 1 min of the end of the hour?
      \item[c.]
        What is the probability that the lecture continues beyond the hour for between 60 and 90 sec?
      \item[d.]
        What is the probability that the lecture continues for at least 90 sec beyond the end of the hour?
    \end{enumerate}
  \item[6.]
    The actual tracking weight of a stereo cartridge that is set to track at 3 g on a particular changer can be regarded as a continuous rv $X$ with pdf
    \begin{align*}
      f(x) = \begin{cases}
        k[1 - (x - 3)^2] & 2 \le x \le 4 \\
        0                & \text{otherwise}
      \end{cases}
    \end{align*}
    \begin{enumerate}
      \item[a.]
        Sketch the graph of $f(x)$.
      \item[b.]
        Find the value of $k$.
      \item[c.]
        What is the probability that the actual tracking weight is greater than the prescribed weight?
      \item[d.]
        What is the probability that the actual weight is within .25 g of the prescribed weight?
      \item[e.]
        What is the probability that the actual weight differs from the prescribed weight by more than .5 g?
    \end{enumerate}
  \end{enumerate}

\end{document}
