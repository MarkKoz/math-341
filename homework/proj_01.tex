\documentclass[letterpaper,12pt]{article}
\usepackage[section=]{mathhw}

\usepackage{tabularray}

\title{MATH 341 - Project 1}

\begin{document}

\maketitle
\begin{enumerate}
  \item[1.]
    The accompanying data set consists of observations on shower-flow rate (L/min) for a sample of houses in Perth, Australia (``An Application of Bayes Methodology to the Analysis of Diary Records in a Water Use Study,'' \textit{J. Amer. Stat. Assoc.}, 1987: 705–711):
    \begin{center}
      \begin{tblr}{colspec = *{10}r}
        4.6 & 12.3 & 7.1 & 7.0 & 4.0 & 9.2 & 6.7 & 6.9 & 11.5 & 5.1 \\
        11.2 & 10.5 & 14.3 & 8.0 & 8.8 & 6.4 & 5.1 & 5.6 & 9.6 & 7.5 \\
        0.2 & 1.6 & 7.5 & 6.2 & 5.8 & 2.3 & 3.4 & 10.4 & 9.8 & 6.6 \\
        3.7 & 6.4 & 8.3 & 6.5 & 7.6 & 9.3 & 9.2 & 7.3 & 5.0 & 6.3 \\
        13.8 & 6.2 & 23.4 & 0.4 & 31.1 & 5.4 & 4.8 & 7.5 & 6.0 & 6.9 \\
        10.8 & 7.5 & 6.6 & 5.0 & 3.3 & 7.6 & 3.9 & 11.9 & 2.2 & 15.0 \\
        7.2 & 6.1 & 15.3 & 18.9 & 7.2 & 26.7 & 5.4 & 5.5 & 4.3 & 9.0 \\
        12.7 & 11.3 & 7.4 & 5.0 & 3.5 & 8.2 & 8.4 & 7.3 & 10.3 & 11.9 \\
        6.0 & 5.6 & 9.5 & 9.3 & 10.4 & 9.7 & 1.2 & 0.8 & 5.1 & 6.7 \\
        10.2 & 6.2 & 8.4 & 7.0 & 4.8 & 5.6 & 10.5 & 14.6 & 10.8 & 15.5 \\
        7.5 & 6.4 & 3.4 & 5.5 & 6.6 & 5.9 & 15.0 & 9.6 & 18.2 & 7.8 \\
        7.0 & 6.9 & 4.1 & 3.6 & 11.9 & 3.7 & 5.7 & 33.1 & 6.8 & 11.3 \\
        9.3 & 9.6 & 10.4 & 9.3 & 6.9 & 9.8 & 9.1 & 10.6 & 4.5 & 6.2 \\
        26.1 & 8.3 & 3.2 & 4.9 & 5.0 & 2.5 & 6.0 & 8.2 & 6.3 & 3.8 \\
        6.0 & 1.5 & 3.1
      \end{tblr}
    \end{center}
    \begin{enumerate}
      \item[1.]
        Draw a stem and leaf plot of these data.
        \begin{center}
          \DefTblrTemplate{caption-tag}{default}{Stem and Leaf Plot 1}
          \begin{longtblr}
          [remark{Key} = {0 | 2 = 0.2}]
          {
            colspec = r|*{17}l,
            rowhead = 1,
            row{1} = {font=\bfseries},
            hline{2} = {solid, leftpos = -1, rightpos = -1, endpos},
            columns = {colsep = 3pt},
            column{1} = {rightsep = 6pt},
            column{2} = {leftsep = 6pt}
          }
            Stem & \multicolumn{17}{l}{Leaf} \\
            0 & 2 & 4 & 8 \\
            1 & 2 & 5 & 6 \\
            2 & 2 & 3 & 5 \\
            3 & 1 & 2 & 3 & 4 & 4 & 5 & 6 & 7 & 7 & 8 & 9 \\
            4 & 0 & 1 & 3 & 5 & 6 & 8 & 8 & 9 \\
            5 & 0 & 0 & 0 & 0 & 1 & 1 & 1 & 4 & 4 & 5 & 5 & 6 & 6 & 6 & 7 & 8 & 9 \\
            6L & 0 & 0 & 0 & 0 & 1 & 2 & 2 & 2 & 2 & 3 & 3 & 4 & 4 & 4 \\
            6H & 5 & 6 & 6 & 6 & 7 & 7 & 8 & 9 & 9 & 9 & 9 \\
            7 & 0 & 0 & 0 & 1 & 2 & 2 & 3 & 3 & 4 & 5 & 5 & 5 & 5 & 5 & 6 & 6 & 8 \\
            8 & 0 & 2 & 2 & 3 & 3 & 4 & 4 & 8 \\
            9 & 0 & 1 & 2 & 2 & 3 & 3 & 3 & 3 & 5 & 6 & 6 & 6 & 7 & 8 & 8 \\
            10 & 2 & 3 & 4 & 4 & 4 & 5 & 5 & 6 & 8 & 8 \\
            11 & 2 & 3 & 3 & 5 & 9 & 9 & 9 \\
            12 & 3 & 7 \\
            13 & 8 \\
            14 & 3 & 6 \\
            15 & 0 & 0 & 3 & 5 \\
            16 \\
            17 \\
            18 & 2 & 9 \\
            19 \\
            20 \\
            21 \\
            22 \\
            23 & 4 \\
            24 \\
            25 \\
            26 & 1 & 7 \\
            27 \\
            28 \\
            29 \\
            30 \\
            31 & 1 \\
            32 \\
            33 & 1
          \end{longtblr}
        \end{center}
      \item[2.]
        Group the data into class intervals and find corresponding frequencies and relative frequencies.
      \item[3.]
        Draw the relative frequencies histogram, polygon, and comment on interesting characteristics (central value, symmetry, modality, variability, and so on).
      \item[4.]
        Draw the cumulative relative frequencies histogram and ogive.
      \item[5.]
        Find 5-number-summary and draw the box plot.
      \item[6.]
        What proportion of the observations in this sample are less than 6?
      \item[7.]
        What proportion of the observations are at least 10?
      \item[8.]
        What proportion of the observations are between 4 and 8 (inclusive)?
    \end{enumerate}
  \item[2.]
    The following data on distilled alcohol content (\%) for a sample of 35 port wines was extracted from the article ``A Method for the Estimation of Alcohol in Fortified Wines Using Hydrometer Baumé and Refractometer Brix'' (\textit{Amer. J. Enol. Vitic.}, 2006: 486–490). Each value is an average of two duplicate measurements.
    \begin{center}
      \begin{tblr}{colspec = *{9}r}
        16.35 & 18.85 & 16.20 & 17.75 & 19.58 & 15.73 & 22.75 & 23.78 & 9.12 \\
        10.22 & 25.25 & 19.08 & 19.62 & 33.05 & 19.20 & 20.05 & 17.85 & 19.17 \\
        19.48 & 20.00 & 19.97 & 42.13 & 31.42 & 03.28 & 41.56 & 17.48 & 17.15 \\
        19.07 & 19.90 & 18.68 & 18.82 & 19.03 & 19.45 & 19.37 & 18.20 & 18.00 \\
        19.60 & 19.33 & 21.22 & 19.50 & 16.30 & 22.25
      \end{tblr}
    \end{center}
    \begin{enumerate}
      \item[a.]
        Draw a stem and leaf plot of these data.
      \item[b.]
        Group the data into class intervals and draw the resulting relative frequency histogram.
      \item[c.]
        Determine the sample mean, median, and mode.
      \item[d.]
        Determine the variance, SD, and range.
      \item[e.]
        Determine quartile 1 (Q1), quartile 2 (Q2), and quartile 3 (Q3).
      \item[f.]
        Determine the IQR and outliers.
      \item[g.]
        Determine the 5-number-summary, draw the corresponding box plot, and comment on interesting characteristics (symmetry, outliers, variability, and so on).
    \end{enumerate}
  \item[3.]
    The table shows daily high temperatures in degrees Fahrenheit recorded over one recent year in Provo, Utah, USA a city far from the ocean and at an elevation of 4500 feet.
    \begin{center}
      \DefTblrTemplate{caption-tag}{default}{Table 3}
      \begin{longtblr}{colspec = *{10}r}
        45 & 46 & 53 & 80 & 72 & 88 & 89 & 70 & 63 & 35 \\
        48 & 49 & 56 & 69 & 78 & 90 & 93 & 58 & 68 & 38 \\
        52 & 50 & 53 & 66 & 88 & 90 & 94 & 44 & 59 & 42 \\
        47 & 49 & 43 & 73 & 93 & 91 & 91 & 61 & 53 & 44 \\
        47 & 39 & 43 & 74 & 92 & 91 & 91 & 68 & 52 & 51 \\
        54 & 45 & 51 & 72 & 92 & 93 & 87 & 73 & 61 & 54 \\
        57 & 46 & 49 & 73 & 93 & 96 & 91 & 76 & 62 & 51 \\
        42 & 45 & 55 & 72 & 98 & 96 & 94 & 77 & 52 & 39 \\
        40 & 39 & 64 & 75 & 95 & 101 & 96 & 79 & 56 & 41 \\
        51 & 31 & 69 & 77 & 85 & 104 & 97 & 80 & 63 & 48 \\
        56 & 40 & 65 & 70 & 82 & 103 & 95 & 84 & 73 & 60 \\
        52 & 38 & 59 & 71 & 82 & 103 & 90 & 84 & 75 & 59 \\
        49 & 35 & 57 & 63 & 89 & 101 & 85 & 81 & 70 & 52 \\
        61 & 35 & 53 & 69 & 94 & 98 & 77 & 76 & 49 & 34 \\
        50 & 35 & 54 & 73 & 92 & 95 & 81 & 76 & 49 & 31 \\
        34 & 44 & 59 & 74 & 80 & 100 & 87 & 74 & 45 & 27 \\
        40 & 51 & 53 & 65 & 73 & 101 & 93 & 65 & 51 & 25 \\
        49 & 55 & 59 & 67 & 74 & 103 & 93 & 64 & 51 & 24 \\
        35 & 59 & 72 & 76 & 86 & 101 & 82 & 64 & 48 & 25 \\
        37 & 62 & 72 & 85 & 92 & 96 & 84 & 65 & 48 & 32 \\
        37 & 67 & 66 & 84 & 96 & 97 & 89 & 59 & 58 & 36 \\
        34 & 60 & 52 & 86 & 86 & 99 & 91 & 62 & 57 & 37 \\
        32 & 54 & 60 & 87 & 88 & 101 & 92 & 67 & 60 & 47 \\
        32 & 62 & 71 & 89 & 88 & 98 & 91 & 72 & 66 & 47 \\
        43 & 64 & 74 & 89 & 91 & 94 & 88 & 66 & 65 & 40 \\
        47 & 55 & 74 & 90 & 94 & 87 & 82 & 68 & 62 & 36 \\
        47 & 63 & 63 & 87 & 92 & 76 & 78 & 66 & 56 & 35 \\
        34 & 62 & 74 & 88 & 94 & 85 & 82 & 52 & 47 & 35 \\
        37 & 58 & 77 & 88 & 93 & 88 & 79 & 52 & 50 & 38 \\
        44 & 45 & 73 & 80 & 95 & 92 & 84 & 59 & 54 & 42 \\
        44 & 45 & 64 & 77 & 94 & 91 & 88 & 60 & 50 & 44 \\
        46 & 38 & 72 & 85 & 93 & 94 & 89 & 56 & 35 & 51 \\
        42 & 39 & 70 & 90 & 92 & 92 & 81 & 58 & 26 &  \\
        48 & 32 & 46 & 90 & 93 & 96 & 76 & 66 & 28 &  \\
        56 & 39 & 56 & 88 & 95 & 94 & 54 & 73 & 36 &  \\
        53 & 54 & 62 & 52 & 91 & 95 & 59 & 68 & 31 &  \\
        42 & 48 & 73 & 67 & 88 & 93 & 67 & 46 & 31
      \end{longtblr}
    \end{center}
    \begin{enumerate}
      \item[1.]
        For given data set determine the number of classes, the class intervals, and the class frequency table.
      \item[2.]
        Draw a relative frequency histogram, polygon, a cumulative relative frequency histogram, and ogive.
      \item[3.]
        Determine the sample mean, median, and mode.
      \item[4.]
        Determine the variance, SD, and range.
      \item[5.]
        What proportion of the observations are between $\bar{x} \pm s$? $\bar{x} \pm 2s$? $\bar{x} \pm 3s$?
      \item[6.]
        Determine quartile 1 (Q1), quartile 2 (Q2), and quartile 3 (Q3).
      \item[7.]
        Determine the IQR and outliers.
      \item[8.]
        Determine the 5-number-summary, draw the corresponding box plot, and comment on interesting characteristics (symmetry, outliers, variability and so on).
    \end{enumerate}
\end{enumerate}

\end{document}
