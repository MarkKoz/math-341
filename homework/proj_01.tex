\documentclass[letterpaper,12pt]{article}
\usepackage[section=]{mathhw}
\usepackage{plotting}
\usepackage{proj_01}

% \usetikzlibrary{plotmarks}

\title{MATH 341 - Project 1}

\begin{document}

\maketitle
\begin{enumerate}
  \item[1.]
    The accompanying data set consists of observations on shower-flow rate (L/min) for a sample of houses in Perth, Australia (``An Application of Bayes Methodology to the Analysis of Diary Records in a Water Use Study,'' \textit{J. Amer. Stat. Assoc.}, 1987: 705–711):
    \begin{center}
      \begin{tblr}{colspec = *{10}r}
        4.6 & 12.3 & 7.1 & 7.0 & 4.0 & 9.2 & 6.7 & 6.9 & 11.5 & 5.1 \\
        11.2 & 10.5 & 14.3 & 8.0 & 8.8 & 6.4 & 5.1 & 5.6 & 9.6 & 7.5 \\
        0.2 & 1.6 & 7.5 & 6.2 & 5.8 & 2.3 & 3.4 & 10.4 & 9.8 & 6.6 \\
        3.7 & 6.4 & 8.3 & 6.5 & 7.6 & 9.3 & 9.2 & 7.3 & 5.0 & 6.3 \\
        13.8 & 6.2 & 23.4 & 0.4 & 31.1 & 5.4 & 4.8 & 7.5 & 6.0 & 6.9 \\
        10.8 & 7.5 & 6.6 & 5.0 & 3.3 & 7.6 & 3.9 & 11.9 & 2.2 & 15.0 \\
        7.2 & 6.1 & 15.3 & 18.9 & 7.2 & 26.7 & 5.4 & 5.5 & 4.3 & 9.0 \\
        12.7 & 11.3 & 7.4 & 5.0 & 3.5 & 8.2 & 8.4 & 7.3 & 10.3 & 11.9 \\
        6.0 & 5.6 & 9.5 & 9.3 & 10.4 & 9.7 & 1.2 & 0.8 & 5.1 & 6.7 \\
        10.2 & 6.2 & 8.4 & 7.0 & 4.8 & 5.6 & 10.5 & 14.6 & 10.8 & 15.5 \\
        7.5 & 6.4 & 3.4 & 5.5 & 6.6 & 5.9 & 15.0 & 9.6 & 18.2 & 7.8 \\
        7.0 & 6.9 & 4.1 & 3.6 & 11.9 & 3.7 & 5.7 & 33.1 & 6.8 & 11.3 \\
        9.3 & 9.6 & 10.4 & 9.3 & 6.9 & 9.8 & 9.1 & 10.6 & 4.5 & 6.2 \\
        26.1 & 8.3 & 3.2 & 4.9 & 5.0 & 2.5 & 6.0 & 8.2 & 6.3 & 3.8 \\
        6.0 & 1.5 & 3.1
      \end{tblr}
    \end{center}
    \begin{enumerate}
      \item[1.]
        Draw a stem and leaf plot of these data.
        \begin{stemleaf}{1}{17}{0 | 2 = 0.2}
          0 & 2 & 4 & 8 \\
          1 & 2 & 5 & 6 \\
          2 & 2 & 3 & 5 \\
          3 & 1 & 2 & 3 & 4 & 4 & 5 & 6 & 7 & 7 & 8 & 9 \\
          4 & 0 & 1 & 3 & 5 & 6 & 8 & 8 & 9 \\
          5 & 0 & 0 & 0 & 0 & 1 & 1 & 1 & 4 & 4 & 5 & 5 & 6 & 6 & 6 & 7 & 8 & 9 \\
          6L & 0 & 0 & 0 & 0 & 1 & 2 & 2 & 2 & 2 & 3 & 3 & 4 & 4 & 4 \\
          6H & 5 & 6 & 6 & 6 & 7 & 7 & 8 & 9 & 9 & 9 & 9 \\
          7 & 0 & 0 & 0 & 1 & 2 & 2 & 3 & 3 & 4 & 5 & 5 & 5 & 5 & 5 & 6 & 6 & 8 \\
          8 & 0 & 2 & 2 & 3 & 3 & 4 & 4 & 8 \\
          9 & 0 & 1 & 2 & 2 & 3 & 3 & 3 & 3 & 5 & 6 & 6 & 6 & 7 & 8 & 8 \\
          10 & 2 & 3 & 4 & 4 & 4 & 5 & 5 & 6 & 8 & 8 \\
          11 & 2 & 3 & 3 & 5 & 9 & 9 & 9 \\
          12 & 3 & 7 \\
          13 & 8 \\
          14 & 3 & 6 \\
          15 & 0 & 0 & 3 & 5 \\
          16 \\
          17 \\
          18 & 2 & 9 \\
          19 \\
          20 \\
          21 \\
          22 \\
          23 & 4 \\
          24 \\
          25 \\
          26 & 1 & 7 \\
          27 \\
          28 \\
          29 \\
          30 \\
          31 & 1 \\
          32 \\
          33 & 1
        \end{stemleaf}
      \item[2.]
        Group the data into class intervals and find corresponding frequencies and relative frequencies.
        \begin{center}
          \begin{tblr}{colspec = >{\itshape}l *{4}c}
            Class              &   0--<3 &   3--<6 &   6--<9 &  9--<12 \\
            Frequency          &       9 &      36 &      50 &      32 \\
            Relative frequency &    .063 &    .252 &    .350 &    .224 \\
            \hline
            Class              & 12--<15 & 15--<18 & 18--<21 & 21--<24 \\
            Frequency          &       5 &       4 &       2 &       1 \\
            Relative frequency &    .035 &    .028 &    .014 &    .007 \\
            \hline
            Class              & 24--<27 & 27--<30 & 30--<33 & 33--<36 \\
            Frequency          &       2 &       0 &       1 &       1 \\
            Relative frequency &    .014 &       0 &    .007 &    .007
          \end{tblr}
        \end{center}
        A class interval of 3 is used. The frequency is the count of the observations which fall within the interval. The relative frequency is the ratio of the frequency to the total number of observations (143). For example, the class interval 15--<18 has 4 observations $\{15.0, 15.0, 15.3, 15.5\}$, so the frequency is 4 and the relative frequency is $\frac{4}{143} \approx 0.028$.
      \item[3.]
        Draw the relative frequencies histogram, polygon, and comment on interesting characteristics (central value, symmetry, modality, variability, and so on).
        \begin{center}
          \begin{tikzpicture}
            \begin{axis}[
              histogram2,
              width = \linewidth,
              xlabel = {Shower Flow Rate (L/min)},
              ylabel = {Relative Frequency},
            ]
              \addplot+[
                hist = {
                  bins = 12,
                  data min = 0,
                  data max = 36,
                },
                y filter/.expression = {y / \dataonenrows},
              ] table[y index = 0] {\dataone};

              % Polygon
              \addplot[
                color = red,
                mark = *,
                mark size = 1pt,
                hist = {
                  bins = 12,
                  data min = 0,
                  data max = 36,
                  handler/.style = {sharp plot},
                  intervals = false,
                },
                y filter/.expression = {y / \dataonenrows},
                shift = {(axis direction cs:1.5,0)},
              ] table[y index = 0] {\dataone};
            \end{axis}
          \end{tikzpicture}
        \end{center}
        The central value is around 7. There is a severe positive skew. The modality is unimodal. It doesn't show high variability since the tall bars are clumped near the central value. Though it is not as evident in the histogram, the last two intervals are outliers (there is only 1 value in each of those intervals). This fact is more obvious when observing the stem and leaf plot.
      \pagebreak
      \item[4.]
        Draw the cumulative relative frequencies histogram and ogive.
        \begin{center}
          \begin{tikzpicture}
            \begin{axis}[
              histogram2,
              width = .9\linewidth,
              xlabel = {Shower Flow Rate (L/min)},
              ylabel = {Cumulative Frequency},
            ]
              \addplot+[
                hist = {
                  cumulative,
                  bins = 12,
                  data min = 0,
                  data max = 36,
                },
                y filter/.expression = {y / \dataonenrows},
              ] table[y index = 0] {\dataone};

              % Ogive
              \addplot[
                color = red,
                mark = *,
                mark size = 1pt,
                hist = {
                  cumulative,
                  bins = 12,
                  data min = 0,
                  data max = 36,
                  handler/.style = {sharp plot},
                  intervals = false,
                },
                y filter/.expression = {y / \dataonenrows},
                shift = {(axis direction cs:3,0)},
              ] table[y index = 0] {\dataone}
              % pos 0 is the start of the prev drawn line; 1 is the end.
              coordinate [pos=0] (a);

              \draw[red] (0, 0) -- (a);
              % https://tex.stackexchange.com/a/64573
              % Disabled becase it clips the x axis.
              % \node[color = red, mark size = 1pt] at (0, 0) {\pgfuseplotmark{*}};
            \end{axis}
          \end{tikzpicture}
        \end{center}
      \item[5.]
        Find 5-number-summary and draw the box plot.
        \begin{center}
          \boxplot{\dataone}{dataone}{width = .9\linewidth, xtick distance = 3}{}
        \end{center}
        \begin{align*}
           \tilde{x} &= x_{\frac{n + 1}{2}} = \dataonemedian \\
           \text{lower quartile} &= \frac{1}{2}(x_{\frac{n + 1}{4}} + x_{\frac{n + 1}{4} + 1}) = \dataonelq \\
           \text{upper quartile} &= \frac{1}{2}(x_{\frac{3(n + 1)}{4}} + x_{\frac{3(n + 1)}{4} + 1}) = \dataoneuq
        \end{align*}
        The interquartile range is determined as follows:
        \begin{align*}
          \text{IQR} &= \text{upper quartile} - \text{lower quartile} \\
          &= \dataoneuq - \dataonelq \\
          &= \eval{\dataoneuq - \dataonelq}
        \end{align*}
        Along with the whisker range of $\dataonewr$, the IQR is used to determine the lower and upper whiskers.
        \begin{align*}
          \text{lower whisker} &= \text{lower quartile} - \dataonewr \cdot IQR \\
          &= \dataonelq - \dataonewr \cdot \eval{\dataoneuq - \dataonelq} \\
          &= \eval{\dataonelq - \dataonewr * (\dataoneuq - \dataonelq)} \\
          \text{upper whisker} &= \text{upper quartile} + \dataonewr \cdot IQR \\
          &= \dataoneuq + \dataonewr \cdot \eval{\dataoneuq - \dataonelq} \\
          &= \eval{\dataoneuq + \dataonewr * (\dataoneuq - \dataonelq)}
        \end{align*}
        The lower whisker is less then the smallest observation ($\dataonelw$), so the lower whisker is simply set to the smallest observation. For the upper whisker, any observation greater than this value is considered an outlier. The largest observation which is below this threshold is $\dataoneuw$, so the upper whisker is set to that value.
      \item[6.]
        What proportion of the observations in this sample are less than 6?
        \begin{align*}
          \frac{9 + 36}{143} = \frac{45}{143} \approx .315
        \end{align*}
        Sum the frequencies of the first two classes (0--<3 and 3--<6), which were found in (2). Then, divide by the total number of observations.
      \item[7.]
        What proportion of the observations are at least 10?
        \begin{align*}
          \frac{(32 - 15) + 5 +4 + 2 + 1 + 2 + 0 + 1 + 1}{143} = \frac{33}{143} \approx .231
        \end{align*}
        Sum the frequencies of the last 8 classes, which were found in (2). Subtract 15 to exclude 9--<10's observations from the 9--<12 class. Then, divide by the total number of observations.
      \item[8.]
        What proportion of the observations are between 4 and 8 (inclusive)?
        \begin{align*}
          \frac{(36 - 11) + 50}{143} = \frac{75}{143} \approx .524
        \end{align*}
        Sum the frequencies of the 3--<6 and 6--<9 classes, which were found in (2). Subtract 11 to exclude 3--<4's observations from the 3--<6 class. Then, divide by the total number of observations.
    \end{enumerate}
  \item[2.]
    The following data on distilled alcohol content (\%) for a sample of 35 port wines was extracted from the article ``A Method for the Estimation of Alcohol in Fortified Wines Using Hydrometer Baumé and Refractometer Brix'' (\textit{Amer. J. Enol. Vitic.}, 2006: 486–490). Each value is an average of two duplicate measurements.
    \begin{center}
      \begin{tblr}{colspec = *{9}r}
        16.35 & 18.85 & 16.20 & 17.75 & 19.58 & 15.73 & 22.75 & 23.78 &  9.12 \\
        10.22 & 25.25 & 19.08 & 19.62 & 33.05 & 19.20 & 20.05 & 17.85 & 19.17 \\
        19.48 & 20.00 & 19.97 & 42.13 & 31.42 &  3.28 & 41.56 & 17.48 & 17.15 \\
        19.07 & 19.90 & 18.68 & 18.82 & 19.03 & 19.45 & 19.37 & 18.20 & 18.00 \\
        19.60 & 19.33 & 21.22 & 19.50 & 16.30 & 22.25
      \end{tblr}
    \end{center}
    \begin{enumerate}
      \item[a.]
        Draw a stem and leaf plot of these data.
        \begin{stemleaf}{2}{15}{15 | 73 = 15.73}
          3 & 28 \\
          4 \\
          5 \\
          6 \\
          7 \\
          8 \\
          9 & 12 \\
          10 & 22 \\
          11 \\
          12 \\
          13 \\
          14 \\
          15 & 73 \\
          16 & 20 & 30 & 35 \\
          17 & 15 & 48 & 75 & 85 \\
          18 & 00 & 20 & 68 & 82 & 85 \\
          19 & 03 & 07 & 08 & 17 & 20 & 33 & 37 & 45 & 48 & 50 & 58 & 60 & 62 & 90 & 97 \\
          20 & 00 & 05 \\
          21 & 22 \\
          22 & 25 & 75 \\
          23 & 78 \\
          24 \\
          25 & 25 \\
          26 \\
          27 \\
          28 \\
          29 \\
          30 \\
          31 & 42 \\
          32 \\
          33 & 05 \\
          34 \\
          35 \\
          36 \\
          37 \\
          38 \\
          39 \\
          40 \\
          41 & 56 \\
          42 & 13
        \end{stemleaf}
      \item[b.]
        Group the data into class intervals and draw the resulting relative frequency histogram.
      \item[c.]
        Determine the sample mean, median, and mode.
      \item[d.]
        Determine the variance, SD, and range.
      \item[e.]
        Determine quartile 1 (Q1), quartile 2 (Q2), and quartile 3 (Q3).
      \item[f.]
        Determine the IQR and outliers.
      \item[g.]
        Determine the 5-number-summary, draw the corresponding box plot, and comment on interesting characteristics (symmetry, outliers, variability, and so on).
    \end{enumerate}
  \item[3.]
    The table shows daily high temperatures in degrees Fahrenheit recorded over one recent year in Provo, Utah, USA a city far from the ocean and at an elevation of 4500 feet.
    \begin{center}
      \DefTblrTemplate{caption-tag}{default}{Table 3}
      \begin{longtblr}{colspec = *{10}r}
        45 & 46 & 53 & 80 & 72 & 88 & 89 & 70 & 63 & 35 \\
        48 & 49 & 56 & 69 & 78 & 90 & 93 & 58 & 68 & 38 \\
        52 & 50 & 53 & 66 & 88 & 90 & 94 & 44 & 59 & 42 \\
        47 & 49 & 43 & 73 & 93 & 91 & 91 & 61 & 53 & 44 \\
        47 & 39 & 43 & 74 & 92 & 91 & 91 & 68 & 52 & 51 \\
        54 & 45 & 51 & 72 & 92 & 93 & 87 & 73 & 61 & 54 \\
        57 & 46 & 49 & 73 & 93 & 96 & 91 & 76 & 62 & 51 \\
        42 & 45 & 55 & 72 & 98 & 96 & 94 & 77 & 52 & 39 \\
        40 & 39 & 64 & 75 & 95 & 101 & 96 & 79 & 56 & 41 \\
        51 & 31 & 69 & 77 & 85 & 104 & 97 & 80 & 63 & 48 \\
        56 & 40 & 65 & 70 & 82 & 103 & 95 & 84 & 73 & 60 \\
        52 & 38 & 59 & 71 & 82 & 103 & 90 & 84 & 75 & 59 \\
        49 & 35 & 57 & 63 & 89 & 101 & 85 & 81 & 70 & 52 \\
        61 & 35 & 53 & 69 & 94 & 98 & 77 & 76 & 49 & 34 \\
        50 & 35 & 54 & 73 & 92 & 95 & 81 & 76 & 49 & 31 \\
        34 & 44 & 59 & 74 & 80 & 100 & 87 & 74 & 45 & 27 \\
        40 & 51 & 53 & 65 & 73 & 101 & 93 & 65 & 51 & 25 \\
        49 & 55 & 59 & 67 & 74 & 103 & 93 & 64 & 51 & 24 \\
        35 & 59 & 72 & 76 & 86 & 101 & 82 & 64 & 48 & 25 \\
        37 & 62 & 72 & 85 & 92 & 96 & 84 & 65 & 48 & 32 \\
        37 & 67 & 66 & 84 & 96 & 97 & 89 & 59 & 58 & 36 \\
        34 & 60 & 52 & 86 & 86 & 99 & 91 & 62 & 57 & 37 \\
        32 & 54 & 60 & 87 & 88 & 101 & 92 & 67 & 60 & 47 \\
        32 & 62 & 71 & 89 & 88 & 98 & 91 & 72 & 66 & 47 \\
        43 & 64 & 74 & 89 & 91 & 94 & 88 & 66 & 65 & 40 \\
        47 & 55 & 74 & 90 & 94 & 87 & 82 & 68 & 62 & 36 \\
        47 & 63 & 63 & 87 & 92 & 76 & 78 & 66 & 56 & 35 \\
        34 & 62 & 74 & 88 & 94 & 85 & 82 & 52 & 47 & 35 \\
        37 & 58 & 77 & 88 & 93 & 88 & 79 & 52 & 50 & 38 \\
        44 & 45 & 73 & 80 & 95 & 92 & 84 & 59 & 54 & 42 \\
        44 & 45 & 64 & 77 & 94 & 91 & 88 & 60 & 50 & 44 \\
        46 & 38 & 72 & 85 & 93 & 94 & 89 & 56 & 35 & 51 \\
        42 & 39 & 70 & 90 & 92 & 92 & 81 & 58 & 26 &  \\
        48 & 32 & 46 & 90 & 93 & 96 & 76 & 66 & 28 &  \\
        56 & 39 & 56 & 88 & 95 & 94 & 54 & 73 & 36 &  \\
        53 & 54 & 62 & 52 & 91 & 95 & 59 & 68 & 31 &  \\
        42 & 48 & 73 & 67 & 88 & 93 & 67 & 46 & 31
      \end{longtblr}
    \end{center}
    \begin{enumerate}
      \item[1.]
        For given data set determine the number of classes, the class intervals, and the class frequency table.
      \item[2.]
        Draw a relative frequency histogram, polygon, a cumulative relative frequency histogram, and ogive.
      \item[3.]
        Determine the sample mean, median, and mode.
      \item[4.]
        Determine the variance, SD, and range.
      \item[5.]
        What proportion of the observations are between $\bar{x} \pm s$? $\bar{x} \pm 2s$? $\bar{x} \pm 3s$?
      \item[6.]
        Determine quartile 1 (Q1), quartile 2 (Q2), and quartile 3 (Q3).
      \item[7.]
        Determine the IQR and outliers.
      \item[8.]
        Determine the 5-number-summary, draw the corresponding box plot, and comment on interesting characteristics (symmetry, outliers, variability and so on).
    \end{enumerate}
\end{enumerate}

\end{document}
