\documentclass[letterpaper,12pt]{article}
\usepackage[section=2.4]{mathhw}
\usepackage{tabularray}

\usepackage{xcolor}
\definecolor{bookblue}{HTML}{00aeef}

\begin{document}

\maketitle

\begin{enumerate}
  \item[45.]
    The population of a particular country consists of three ethnic groups. Each individual belongs to one of the four major blood groups. The accompanying \textit{joint probability table} gives the proportions of individuals in the various ethnic group–blood group combinations.
    \begin{center}
      \begin{tblr}{
        colspec = *{2}{>{\bfseries}l}*{4}{c},
        row{1,2} = {font=\bfseries},
        hline{1,3,6} = {0.7pt, solid, bookblue, leftpos = -1, rightpos = -1, endpos, abovespace = 0.33em, belowspace = 0.33em},
        hline{2} = {2-6}{0.7pt, solid, bookblue, leftpos = -1, rightpos = -1, endpos, abovespace = 0.3em, belowspace = 0.33em},
      }
        & \SetCell[c=5]{c} Blood Group \\
        &   & O    & A    & B    & AB \\
        \SetCell[r=3]{l} Ethnic Group
        & 1 & .082 & .106 & .008 & .004 \\
        & 2 & .135 & .141 & .018 & .006 \\
        & 3 & .215 & .200 & .065 & .020 \\
      \end{tblr}
    \end{center}
    Suppose that an individual is randomly selected from the population, and define events by $A = \{$type A selected$\}$, $B = \{$type B selected$\}$, and $C = \{$ethnic group 3 selected$\}$.
    \begin{enumerate}
      \item[a.]
        Calculate $P(A)$, $P(C)$, and $P(A \cap C)$.
        \begin{align*}
          P(A) &= P((A \cap E_1) \cup (A \cap E_2) \cup (A \cap E_3)) \\
          &= P(A \cap E_1) + P(A \cap E_2) + P(A \cap E_3) \\
          &= .106 + .141 + .200 \\
          &= .447
        \end{align*}
        \begin{align*}
          P(C) &= P((T_O \cap E_3) \cup (T_A \cap E_3) \cup (T_B \cap E_3) \cup (T_{AB} \cap E_3)) \\
          &= P(T_{O} \cap E_3) + P(T_{A} \cap E_3) + P(T_{B} \cap E_3) + P(T_{AB} \cap E_3) \\
          &= .215 + .200 + .065 + .020 \\
          &= .500
        \end{align*}
        \begin{align*}
          P(A \cap C) &= .200
        \end{align*}
      \item[b.]
        Calculate both $P(A|C)$ and $P(C|A)$, and explain in context what each of these probabilities represents.
        \begin{align*}
          P(A|C) &= \frac{P(A \cap C)}{P(C)} = \frac{.200}{.500} = .400
        \end{align*}
        $P(A|C)$ is the probability of blood group $A$ being selected given that ethnic group 3 was selected.
        \begin{align*}
          P(C|A) &= \frac{P(C \cap A)}{P(A)} = \frac{.200}{.447} \approx .447
        \end{align*}
        $P(C|A)$ is the probability of ethnic group 3 being selected given that blood group $A$ was selected.
      \item[c.]
        If the selected individual does not have type B blood, what is the probability that he or she is from ethnic group 1?
        \begin{align*}
          P(B^\prime) &= 1 - P(B) \\
          &= 1 - P((B \cap E_1) \cup (B \cap E_2) \cup (B \cap E_3)) \\
          &= 1 - P(B \cap E_1) + P(B \cap E_2) + P(B \cap E_3) \\
          &= 1 - .008 + .018 + .065 \\
          &= 1 - .091 \\
          &= .909
        \end{align*}
        \begin{align*}
          P(E_1|B^\prime) &= \frac{P(E_1 \cap B^\prime)}{P(B^\prime)} \\
          &= \frac{P(E_1 \cap (T_O \cup T_A \cup T_{AB}))}{P(B^\prime)} \\
          &= \frac{P((E_1 \cap T_O) \cup (E_1 \cap T_A) \cup (E_1 \cap T_{AB})}{P(B^\prime)} \\
          &= \frac{P(E_1 \cap T_O) + P(E_1 \cap T_A) + P(E_1 \cap T_{AB})}{P(B^\prime)} \\
          &= \frac{.082 + .106 + .004}{.909} \\
          &= \frac{.192}{.909} \\
          &\approx .211
        \end{align*}
    \end{enumerate}
  \item[49.]
    The accompanying table gives information on the type of coffee selected by someone purchasing a single cup at a particular airport kiosk.
    \begin{center}
      \begin{tblr}{
        colspec = l*{3}{c},
        row{1} = {font=\bfseries},
        hline{1,2,4} = {0.7pt, solid, bookblue, leftpos = -1, rightpos = -1, endpos, abovespace = 0.33em, belowspace = 0.33em},
      }
                & Small & Medium & Large \\
        Regular & 14\% & 20\% & 26\% \\
        Decaf   & 20\% & 10\% & 10\%
      \end{tblr}
    \end{center}
    Consider randomly selecting such a coffee purchaser. Let $S_s$, $S_m$, and $S_l$ denote the event in which a small, medium, or large cup, respectively, is purchased. Let $T_r$ and $T_d$ denote the event in which a regular or decafe cup, respectively, is purchased.
    \begin{enumerate}
      \item[a.]
        What is the probability that the individual purchased a small cup? A cup of decaf coffee?
        \begin{align*}
          P(S_s) &= P(S_s \cap (T_r \cup T_d)) \\
          &= P((S_s \cap T_r) \cup (S_s \cap T_d)) \\
          &= P(S_s \cap T_r) + P(S_s \cap T_d) \\
          &= .14 + .20 \\
          &= .34
        \end{align*}
        \begin{align*}
          P(T_d) &= P((S_s \cup S_m \cup S_l) \cap T_d) \\
          &= P((S_s \cap T_d) \cup (S_m \cap T_d) \cup (S_l \cap T_d)) \\
          &= P(S_s \cap T_d) + P(S_m \cap T_d) + P(S_l \cap T_d) \\
          &= .20 + .10 + .10 \\
          &= .40
        \end{align*}
      \item[b.]
        If we learn that the selected individual purchased a small cup, what now is the probability that he/she chose decaf coffee, and how would you interpret this probability?
        \begin{align*}
          P(T_d|S_s) &= \frac{P(T_d \cap S_s)}{P(S_s)} = \frac{.20}{.34} = .588
        \end{align*}
        This means 58.8\% of small coffee cups bought at the kiosk contain decaf coffee.
      \item[c.]
        If we learn that the selected individual purchased decaf, what now is the probability that a small size was selected, and how does this compare to the corresponding unconditional probability of (a)?
        \begin{align*}
          P(S_s|T_d) &= \frac{P(S_s \cap T_d)}{P(T_d)} = \frac{.20}{.40} = .50
        \end{align*}
        This probability is larger than the one found in (a). 50\% of bought decaf cups are small while 34\% of cups containing either coffee type are small.
    \end{enumerate}
  \item[53.]
    A certain shop repairs both audio and video components. Let $A$ denote the event that the next component brought in for repair is an audio component, and let $B$ be the event that the next component is a compact disc player (so the event $B$ is contained in $A$). Suppose that $P(A) = .6$ and $P(B) = .05$. What is $P(B|A)$?
    \begin{align*}
      P(B|A) = \frac{P(B \cap A)}{P(A)} = \frac{P(B)}{P(A)} = \frac{.05}{.6} = 0.08\bar{3}
    \end{align*}
    Because $B \subset A$, $B \cap A = B$. That is, $B$ has all of its elements in common with $A$.
  \item[59.]
    At a certain gas station, 40\% of the customers use regular gas ($A_1$), 35\% use plus gas ($A_2$), and 25\% use premium ($A_3$). Of those customers using regular gas, only 30\% fill their tanks (event $B$). Of those customers using plus, 60\% fill their tanks, whereas of those using premium, 50\% fill their tanks.
    \begin{enumerate}
      \item[a.]
        What is the probability that the next customer will request plus gas and fill the tank ($A_2 \cap B$)?
      \item[b.]
        What is the probability that the next customer fills the tank?
      \item[c.]
        If the next customer fills the tank, what is the probability that regular gas is requested? Plus? Premium?
    \end{enumerate}
  \item[62.]
    Blue Cab operates 15\% of the taxis in a certain city, and Green Cab operates the other 85\%. After a nighttime hit-and-run accident involving a taxi, an eyewitness said the vehicle was blue. Suppose, though, that under night vision conditions, only 80\% of individuals can correctly distinguish between a blue and a green vehicle. What is the (posterior) probability that the taxi at fault was blue? In answering, be sure to indicate which probability rules you are using. [\textit{Hint:} A tree diagram might help. \textit{Note:} This is based on an actual incident.]
\end{enumerate}

\end{document}
