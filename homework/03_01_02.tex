\documentclass[letterpaper,12pt]{article}
\usepackage[section=3.1\ \&\ 3.2]{mathhw}
\usepackage{plotting}
\usepackage{tabularray}
\usepackage{braket}
\usepackage{mathtools}

% https://tex.stackexchange.com/a/394772
\newcommand{\equationinalign}[2]{%
  \multispan{#2}%
  \hfill$\displaystyle{#1}$\hfill
  \ignorespaces
}

\begin{document}

\maketitle

\begin{enumerate}
  \item[1.]
    A concrete beam may fail either by shear ($S$) or flexure ($F$). Suppose that three failed beams are randomly selected and the type of failure is determined for each one. Let $X =$ the number of beams among the three selected that failed by shear. List each outcome in the sample space along with the associated value of $X$.
    \begin{align*}
      \mathcal{S} &= \mathrlap{\Set{SSS, FFF, SSF, SFS, FSS, FFS, FSF, SFF}} & & & & \\
      X(SSS) &= 3 & X(SSF) &= 2 & X(FFS) &= 1 \\
      X(FFF) &= 0 & X(SFS) &= 2 & X(FSF) &= 1 \\
                & & X(FSS) &= 2 & X(SFF) &= 1
    \end{align*}
  \item[4.]
    Let $X =$ the number of nonzero digits in a randomly selected 4-digit PIN that has no restriction on the digits. What are the possible values of $X$? Give three possible outcomes and their associated $X$ values.
    \begin{align*}
      \equationinalign{x = \Set{0, 1, 2, 3, 4}}{6} \\
      X(1234) &= 4 & X(7014) &= 3 & X(9009) &= 2
    \end{align*}
  \item[11.]
    Let $X$ be the number of students who show up for a professor’s office hour on a particular day. Suppose that the pmf of $X$ is $p(0) = .20$, $p(1) = .25$, $p(2) = .30$, $p(3) = .15$, and $p(4) = .10$.
    \begin{enumerate}
      \item[a.]
        Draw the corresponding probability histogram.
        \begin{center}
          \begin{tikzpicture}
            \begin{axis}[
              histogram2,
              xlabel = Students,
              ylabel = Probability,
              width = .5\linewidth,
            ]
              \addplot+[
                ybar interval,
              ] plot coordinates {
                (0, .20) (1, .25) (2, .30) (3, .15) (4, .10)
              };
            \end{axis}
          \end{tikzpicture}
        \end{center}
      \item[b.]
        What is the probability that at least two students show up? More than two students show up?
        \begin{align*}
          p(x \ge 2) &= p(2) + p(3) + p(4) = .30 + .15 + .10 = .55 \\
          p(x > 2) &= p(x \ge 2) - p(2) = .55 - .30 = .25
        \end{align*}
      \item[c.]
        What is the probability that between one and three students, inclusive, show up?
        \begin{align*}
          p(1 \le x \le 3) = p(1) + p(2) + p(3) = .25 + .30 + .15 = .70
        \end{align*}
      \item[d.]
        What is the probability that the professor shows up?
        \\ \\
        Cannot be determined due to insufficient information. The given pmf only concerns the students' attendance.
    \end{enumerate}
  \item[13.]
    A mail-order computer business has six telephone lines. Let $X$ denote the number of lines in use at a specified time. Suppose the pmf of $X$ is as given in the accompanying table.
    \begin{center}
      \begin{tblr}{colspec = >{$}l<{$}|*{7}c}
        x & 0 & 1 & 2 & 3 & 4 & 5 & 6 \\
        \hline
        p(x) & .10 & .15 & .20 & .25 & .20 & .06 & .04
      \end{tblr}
    \end{center}
    Calculate the probability of each of the following events.
    \begin{enumerate}
      \item[a.]
        \{at most three lines are in use\}
      \item[b.]
        \{fewer than three lines are in use\}
      \item[c.]
        \{at least three lines are in use\}
      \item[d.]
        \{between two and five lines, inclusive, are in use\}
      \item[e.]
        \{between two and four lines, inclusive, are not in use\}
      \item[f.]
        \{at least four lines are not in use\}
    \end{enumerate}
  \item[16.]
    Some parts of California are particularly earthquake-prone. Suppose that in one metropolitan area, 25\% of all homeowners are insured against earthquake damage. Four homeowners are to be selected at random; let $X$ denote the number among the four who have earthquake insurance.
    \begin{enumerate}
      \item[a.]
        Find the probability distribution of X. [\textit{Hint:} Let $S$ denote a homeowner who has insurance and $F$ one who does not. Then one possible outcome is $SFSS$, with probability (.25)(.75)(.25)(.25) and associated $X$ value 3. There are 15 other outcomes.]
      \item[b.]
        Draw the corresponding probability histogram.
      \item[c.]
        What is the most likely value for $X$?
      \item[d.]
        What is the probability that at least two of the four selected have earthquake insurance?
      \end{enumerate}
  \item[18.]
    Two fair six-sided dice are tossed independently. Let $M =$ the maximum of the two tosses (so $M(1,5) = 5$, $M(3,3) = 3$, etc.).
    \begin{enumerate}
      \item[a.]
        What is the pmf of $M$? [\textit{Hint:} First determine $p(1)$, then $p(2)$, and so on.]
      \item[b.]
        Determine the cdf of $M$ and graph it.
    \end{enumerate}
\end{enumerate}

\end{document}
