\documentclass[letterpaper,12pt]{article}
\usepackage[section=4.2]{mathhw}
\usepackage{physics}
\usepackage[makeroom]{cancel}

\begin{document}

\maketitle

\begin{enumerate}
  \item[11.]
    Let $X$ denote the amount of time a book on two-hour reserve is actually checked out, and suppose the cdf is
    \begin{align*}
      F(x) = \begin{cases}
        0             & x < 0 \\
        \frac{x^2}{4} & 0 \le x < 2 \\
        1             & 2 \le x
      \end{cases}
    \end{align*}
    \begin{enumerate}
      \item[a.]
        Calculate $P(X \le 1)$.
        \begin{align*}
          P(X \le 1) = F(1) = \frac{1^2}{4} = \frac{1}{4}
        \end{align*}
      \item[b.]
        Calculate $P(.5 \le X \le 1)$.
        \begin{align*}
          P(.5 \le X \le 1) &= F(1) - F(.5) \\
          &= \frac{1^2}{4} - \frac{.5^2}{4} \\
          &= \frac{1}{4} - \frac{1}{16} \\
          &= \frac{3}{16}
        \end{align*}
      \item[c.]
        Calculate $P(X > 1.5)$.
        \begin{align*}
          P(X > 1.5) &= 1 - P(X \le 1.5) \\
          &= 1 - F(1.5) \\
          &= 1 - \frac{1.5^2}{4} \\
          &= 1 - \frac{9}{16} \\
          &= \frac{7}{16}
        \end{align*}
      \item[d.]
        What is the median checkout duration $\tilde{\mu}$? [solve $.5 = F(\tilde{\mu})$].
        \begin{align*}
          .5 &= F(\tilde{\mu}) = \frac{\tilde{\mu}^2}{4} \\
          2 &= \tilde{\mu}^2 \\
          \sqrt{2} &= \tilde{\mu}
        \end{align*}
      \item[e.]
        Obtain the density function $f(x)$.
        \begin{align*}
          f(x) &= F^\prime(x) \\
          &= \dv{x}(\frac{x^2}{4}) \\
          &= \frac{1}{4} \cdot \dv{x^2}{x} \\
          &= \frac{1}{4} \cdot 2x \\
          &= \frac{x}{2}
        \end{align*}
      \item[f.]
        Calculate $E(X)$.
        \begin{align*}
          E(X) &= \int_{0}^{2} x \cdot \frac{x}{2} \dd{x} \\
          &= \frac{1}{2} \int_{0}^{2} x^2 \dd{x} \\
          &= \frac{1}{2} \cdot \frac{x^3}{3} \bigg\rvert_{0}^{2} \\
          &= \frac{1}{6}(2^3 - 0^3) \\
          &= \frac{4}{3}
        \end{align*}
      \item[g.]
        Calculate $V(X)$ and $\sigma_X$.
        \begin{align*}
          E(X^2) &= \int_{0}^{2} x^2 \cdot \frac{x}{2} \dd{x} \\
          &= \frac{1}{2} \int_{0}^{2} x^3 \dd{x} \\
          &= \frac{1}{2} \cdot \frac{x^4}{4} \bigg\rvert_{0}^{2} \\
          &= \frac{1}{8}(2^4 - 0^3) \\
          &= 2 \\
          \\
          V(X) &= E(X^2) - [E(X)]^2 \\
          &= 2 - \left(\frac{4}{3}\right)^2 \\
          &= \frac{18}{9} - \frac{16}{9} \\
          &= \frac{2}{9} \\
          \\
          \sigma_X &= \sqrt{V(X)} = \sqrt{\frac{2}{9}} = \frac{\sqrt{2}}{3}
        \end{align*}
      \item[h.]
        If the borrower is charged an amount $h(X) = X^2$ when checkout duration is $X$, compute the expected charge $E[h(X)]$.
        \begin{align*}
          E(X^2) = 2
        \end{align*}
        Same as the calculation in (g).
    \end{enumerate}
  \item[12.]
    The cdf for $X$ (= measurement error) of Exercise 3 is
    \begin{align*}
      F(x) = \begin{cases}
        0                                                         & x < -2 \\
        \frac{1}{2} + \frac{3}{32}\left(4x - \frac{x^3}{3}\right) & -2 \le x < 2 \\
        1                                                         & 2 \le x
      \end{cases}
    \end{align*}
    \begin{enumerate}
      \item[a.]
        Calculate $P(X < 0)$.
        \begin{align*}
          P(X < 0) &= F(0) \\
          &= \frac{1}{2} + \frac{3}{32}\left(4(0) - \frac{0^3}{3}\right) \\
          &= \frac{1}{2} + \frac{3}{32}(0) \\
          &= \frac{1}{2}
        \end{align*}
      \item[b.]
        Calculate $P(-1 < X < 1)$.
        \begin{align*}
          P(-1 < X < 1) &= F(1) - F(-1) \\
          &= \cancel{\frac{1}{2}} + \frac{3}{32}\left(4(1) - \frac{1^3}{3}\right) \cancel{- \frac{1}{2}} - \frac{3}{32}\left(4(-1) - \frac{(-1)^3}{3}\right) \\
          &= \frac{3}{32}\left(4 - \frac{1}{3}\right) - \frac{3}{32}\left(-4 + \frac{1}{3}\right) \\
          &= \frac{\cancel{3}}{32} \cdot \frac{11}{\cancel{3}} + \frac{\cancel{3}}{32} \cdot \frac{11}{\cancel{3}} \\
          &= \frac{11}{32} + \frac{11}{32} \\
          &= \frac{11}{16}
        \end{align*}
      \item[c.]
        Calculate $P(.5 < X)$.
        \begin{align*}
          P(.5 < X) &= 1 - P(X \le .5) \\
          &= 1 - F(.5) \\
          &= 1 - \frac{1}{2} - \frac{3}{32}\left(4 \cdot \frac{1}{2} - \frac{1}{3} \cdot \frac{1^3}{2^3}\right) \\
          &= \frac{1}{2} - \frac{3}{32}\left(2 - \frac{1}{24}\right) \\
          &= \frac{1}{2} - \frac{3}{32} \cdot \frac{47}{24} \\
          &= \frac{81}{256} \\
        \end{align*}
      \item[d.]
        Verify that $f(x)$ is as given in Exercise 3 by obtaining $F^\prime(x)$.
        \begin{align*}
          f(x) &= F^\prime(x) \\
          &= \dv{x}(\frac{1}{2} + \frac{3}{32}\left(4x - \frac{x^3}{3}\right)) \\
          &= \dv{x}(\frac{1}{2}) + \frac{3}{32}\left(4 \cdot \dv{x}{x} - \frac{1}{3} \cdot \dv{x^3}{x}\right) \\
          &= 0 + \frac{3}{32}\left(4 \cdot 1 - \frac{1}{3} \cdot 3x^2 \right) \\
          &= \frac{3}{32}(4 - x^2) \\
          &= .09375(4 - x^2)
        \end{align*}
      \item[e.]
        Verify that $\tilde{\mu} = 0$.
        \begin{align*}
          .5 &= F(\tilde{\mu}) = \frac{1}{2} + \frac{3}{32}\left(4\tilde{\mu} - \frac{\tilde{\mu}^3}{3}\right) \\
          0 &= \frac{3}{32}\left(4\tilde{\mu} - \frac{\tilde{\mu}^3}{3}\right) \\
          0 &= 4\tilde{\mu} - \frac{\tilde{\mu}^3}{3} \\
          0 &= -\frac{\tilde{\mu}}{3}(\tilde{\mu}^2 - 12) \\
          0 &= \tilde{\mu}(\tilde{\mu}^2 - 12)
        \end{align*}
        \begin{align*}
          \tilde{\mu} = 0\ \text{or}\ \tilde{\mu}^2 - 12 = 0 \rightarrow \tilde{\mu} = \sqrt{12}
        \end{align*}
        $\sqrt{12}$ is not in the interval $[2,-2]$, so $\tilde{\mu} = 0$ is the only valid solution.
    \end{enumerate}
  \item[13.]
    Example 4.5 introduced the concept of time headway in traffic flow and proposed a particular distribution for $X =$ the headway between two randomly selected consecutive cars (sec). Suppose that in a different traffic environment, the distribution of time headway has the form
    \begin{align*}
      f(x) = \begin{cases}
        \frac{k}{x^4} & x > 1 \\
        0             & x \le 1
      \end{cases}
    \end{align*}
    \begin{enumerate}
      \item[a.]
        Determine the value of $k$ for which $f(x)$ is a legitimate pdf.
        \begin{align*}
          1 &= \int_{-\infty}^{\infty} f(x) \dd{x} \\
          &= \int_{-\infty}^{1} f(x) \dd{x} + \int_{1}^{\infty} f(x) \dd{x} \\
          &= 0 + \int_{1}^{\infty} \frac{k}{x^4} \dd{x} \\
          &= k \int_{1}^{\infty} x^{-4} \dd{x} \\
          &= k \lim_{b \to \infty} \left(-\frac{1}{3x^3}\right) \bigg\rvert_{1}^{b} \\
          &= -\frac{k}{3} \cdot \frac{1}{(\lim_{b \to \infty} x)^3} \bigg\rvert_{1}^{b} \\
          &= -\frac{k}{3} \left[\frac{1}{(\lim_{b \to \infty} b)^3} - \frac{1}{(\lim_{b \to \infty} 1)^3}\right] \\
          &= -\frac{k}{3} \left[\frac{1}{\infty^3} - \frac{1}{1^3}\right] \\
          &= -\frac{k}{3} [0 - 1] \\
          &= \frac{k}{3} \\
          \\
          3 &= k
        \end{align*}
      \item[b.]
        Obtain the cumulative distribution function.
        \begin{align*}
          F(X) &= \int_{-\infty}^{1} f(x) \dd{x} + \int_{1}^{x} f(x) \dd{x} \\
          &= 0 + \int_{1}^{x} \frac{3}{x^4} \dd{x} \\
          &= 3 \int_{1}^{x} x^{-4} \dd{x} \\
          &= 3 \cdot -\frac{1}{3x^3} \bigg\rvert_{1}^{x} \\
          &= -\frac{1}{x^3} + \frac{1}{1^3} \\
          &= 1 - \frac{1}{x^3}
        \end{align*}
      \item[c.]
        Use the cdf from (b) to determine the probability that headway exceeds 2 sec and also the probability that headway is between 2 and 3 sec.
        \begin{align*}
          P(X > 2) &= 1 - P(X \le 2) \\
          &= 1 - F(2) \\
          &= 1 - \left(1 - \frac{1}{2^3}\right) \\
          &= \frac{1}{8}
        \end{align*}
        \begin{align*}
          P(2 \le X \le 3) &= F(3) - F(2) \\
          &= \left(1 - \frac{1}{3^3}\right) - \left(1 - \frac{1}{2^3}\right) \\
          &= 1 - \frac{1}{27} - 1 + \frac{1}{8} \\
          &= \frac{19}{216}
        \end{align*}
      \item[d.]
        Obtain the mean value of headway and the standard deviation of headway.
        \begin{align*}
          \mu_X &= \int_{-\infty}^{\infty} x \cdot f(x) \dd{x} \\
          &= \int_{-\infty}^{1} x \cdot f(x) \dd{x} + \int_{1}^{\infty} x \cdot f(x) \dd{x} \\
          &= 0 + \int_{1}^{\infty} x \cdot \frac{3}{x^4} \dd{x} \\
          &= 3 \int_{1}^{\infty} x^{-3} \dd{x} \\
          &= 3 \lim_{b \to \infty} \left(-\frac{1}{2x^2}\right) \bigg\rvert_{1}^{b} \\
          &= -\frac{3}{2} \cdot \frac{1}{(\lim_{b \to \infty} x)^2} \bigg\rvert_{1}^{b} \\
          &= -\frac{3}{2} \left[\frac{1}{(\lim_{b \to \infty} b)^2} - \frac{1}{(\lim_{b \to \infty} 1)^2}\right] \\
          &= -\frac{3}{2} \left[\frac{1}{\infty^2} - \frac{1}{1^2}\right] \\
          &= -\frac{3}{2} [0 - 1] \\
          &= \frac{3}{2}
        \end{align*}
        \begin{align*}
          \mu_{X^2} &= \int_{-\infty}^{\infty} x^2 \cdot f(x) \dd{x} \\
          &= \int_{-\infty}^{1} x^2 \cdot f(x) \dd{x} + \int_{1}^{\infty} x^2 \cdot f(x) \dd{x} \\
          &= 0 + \int_{1}^{\infty} x^2 \cdot \frac{3}{x^4} \dd{x} \\
          &= 3 \int_{1}^{\infty} x^{-2} \dd{x} \\
          &= 3 \lim_{b \to \infty} \left(-\frac{1}{x}\right) \bigg\rvert_{1}^{b} \\
          &= -3 \cdot \frac{1}{\lim_{b \to \infty} x} \bigg\rvert_{1}^{b} \\
          &= -3 \left[\frac{1}{\lim_{b \to \infty} b} - \frac{1}{\lim_{b \to \infty} 1}\right] \\
          &= -3 \left[\frac{1}{\infty} - \frac{1}{1}\right] \\
          &= -3 [0 - 1] \\
          &= 3
        \end{align*}
        \begin{align*}
          \sigma_X &= \sqrt{\mu_{X^2} - (\mu_X)^2} \\
          &= \sqrt{3 - \left(\frac{3}{2}\right)^2} \\
          &= \sqrt{\frac{12}{4} - \frac{9}{4}} \\
          &= \frac{\sqrt{3}}{2}
        \end{align*}
      \item[e.]
        What is the probability that headway is within 1 standard deviation of the mean value?
        \begin{align*}
          P(\mu - \sigma \le X \le \mu + \sigma) &= P\left(\frac{3 - \sqrt{3}}{2} \le X \le \frac{3 + \sqrt{3}}{2}\right) \\
          &= F\left(\frac{3 + \sqrt{3}}{2}\right) - F\left(\frac{3 - \sqrt{3}}{2}\right) \\
          &= \left(1 - \frac{1}{\left(\frac{3 + \sqrt{3}}{2}\right)^3}\right) - \left(1 - \frac{1}{\left(\frac{3 - \sqrt{3}}{2}\right)^3}\right) \\
          &= \frac{8}{(3 - \sqrt{3})^3} - \frac{8}{(3 + \sqrt{3})^3} \\
          &= \frac{8}{6(9 - 5\sqrt{3})} - \frac{8}{6(9 + 5\sqrt{3})} \\
          &= \frac{4(9 + 5\sqrt{3})}{3(9 - 5\sqrt{3})(9 + 5\sqrt{3})} - \frac{4(9 - 5\sqrt{3})}{3(9 + 5\sqrt{3})(9 - 5\sqrt{3})} \\
          &= \frac{4(9 + 5\sqrt{3})}{3 \cdot 6} - \frac{4(9 - 5\sqrt{3})}{3 \cdot 6} \\
          &= \frac{2(9 + 5\sqrt{3})}{9} - \frac{2(9 - 5\sqrt{3})}{9} \\
          &= \frac{18 + 10\sqrt{3}}{9} - \frac{18 - 10\sqrt{3}}{9} \\
          &= \frac{20\sqrt{3}}{9}
        \end{align*}
    \end{enumerate}
  \item[15.]
    Let $X$ denote the amount of space occupied by an article placed in a 1-ft$^3$ packing container. The pdf of $X$ is
    \begin{align*}
      f(x) = \begin{cases}
        90x^8(1 - x) & 0 < x < 1 \\
        0            & \text{otherwise}
      \end{cases}
    \end{align*}
    \begin{enumerate}
      \item[a.]
        Graph the pdf. Then obtain the cdf of $X$ and graph it.
        \begin{align*}
          F(X) &= \int_{-\infty}^{x} f(x) \dd{x} \\
          &= \int_{0}^{x} 90x^8(1 - x) \dd{x} \\
          &= 90 \left(\int_{0}^{x} x^8 \dd{x} - \int_{0}^{x} x^9 \dd{x}\right) \\
          &= 90 \left(\frac{x^9}{9} \bigg\rvert_{0}^{x} - \frac{x^{10}}{10} \bigg\rvert_{0}^{x}\right) \\
          &= 90 \left(\frac{1}{9}(x^9 - 0^9) - \frac{1}{10}(x^{10} - 0^{10})\right) \\
          &= 90 \left(\frac{x^9}{9} - \frac{x^{10}}{10}\right) \\
          &= 10x^9 - 9x^{10}
        \end{align*}
      \item[b.]
        What is $P(X \le .5)$ [i.e., $F(.5)$]?
        \begin{align*}
          F(.5) &= 10\left(\frac{1}{2}\right)^9 - 9\left(\frac{1}{2}\right)^{10} \\
          &= \frac{10}{512} - \frac{9}{1024} \\
          &= \frac{11}{1024}
        \end{align*}
      \item[c.]
        Using the cdf from (a), what is $P(.25 < X \le .5)$? What is $P(.25 \le X \le .5)$?
        \begin{align*}
          P(.25 < X \le .5) &= F(.5) - F(.25) \\
          &= \frac{11}{1024} - \left[10\left(\frac{1}{4}\right)^9 - 9\left(\frac{1}{4}\right)^{10}\right] \\
          &= \frac{11}{1024} - \left[\frac{10}{262144} - \frac{9}{1048576}\right] \\
          &= \frac{11233}{1048576}
        \end{align*}
      \item[d.]
        What is the 75th percentile of the distribution?
        \begin{align*}
          .75 &= F(\eta(.75)) = 10(\eta(.75))^9 - 9(\eta(.75))^{10} \\
          \eta(.75) &\approx .903596
        \end{align*}
      \item[e.]
        Compute $E(X)$ and $\sigma_X$.
        \begin{align*}
          E(X) &= \int_{-\infty}^{\infty} x \cdot f(x) \dd{x} \\
          &= \int_{0}^{1} x \cdot 90x^8(1 - x) \dd{x} \\
          &= 90 \left(\int_{0}^{1} x^9 \dd{x} - \int_{0}^{1} x^{10} \dd{x}\right) \\
          &= 90 \left(\frac{x^{10}}{10} \bigg\rvert_{0}^{1} - \frac{x^{11}}{11} \bigg\rvert_{0}^{1}\right) \\
          &= 90 \left(\frac{1}{10}(1^{10} - 0^{10}) - \frac{1}{11}(1^{11} - 0^{11})\right) \\
          &= 90 \left(\frac{1}{10} - \frac{1}{11}\right) \\
          &= \frac{9}{11}
        \end{align*}
        \begin{align*}
          E(X^2) &= \int_{-\infty}^{\infty} x^2 \cdot f(x) \dd{x} \\
          &= \int_{0}^{1} x^2 \cdot 90x^8(1 - x) \dd{x} \\
          &= 90 \left(\int_{0}^{1} x^{10} \dd{x} - \int_{0}^{1} x^{11} \dd{x}\right) \\
          &= 90 \left(\frac{x^{11}}{11} \bigg\rvert_{0}^{1} - \frac{x^{12}}{12} \bigg\rvert_{0}^{1}\right) \\
          &= 90 \left(\frac{1}{11}(1^{11} - 0^{11}) - \frac{1}{12}(1^{12} - 0^{12})\right) \\
          &= 90 \left(\frac{1}{11} - \frac{1}{12}\right) \\
          &= \frac{15}{22}
        \end{align*}
        \begin{align*}
          \sigma_X &= \sqrt{E(X^2) - [E(X)]^2} \\
          &= \sqrt{\frac{15}{22} - \left(\frac{9}{11}\right)^2} \\
          &= \sqrt{\frac{15}{22} - \frac{81}{121}} \\
          &= \sqrt{\frac{3}{242}} \\
          &= \frac{\sqrt{6}}{22}
        \end{align*}
      \item[f.]
        What is the probability that $X$ is more than 1 standard deviation from its mean value?
        \begin{align*}
          1 - P(\mu - \sigma \le X \le \mu + \sigma) &= 1 - P\left(\frac{9}{11} - \frac{\sqrt{6}}{22} \le X \le \frac{9}{11} + \frac{\sqrt{6}}{22}\right) \\
          &= 1 - F\left(\frac{9}{11} + \frac{\sqrt{6}}{22}\right) + F\left(\frac{9}{11} - \frac{\sqrt{6}}{22}\right) \\
          &= \begin{aligned}[t]
            1 &- \left[10\left(\frac{9}{11} + \frac{\sqrt{6}}{22}\right)^9 - 9\left(\frac{9}{11} + \frac{\sqrt{6}}{22}\right)^{10}\right] \\
            &+ \left[10\left(\frac{9}{11} - \frac{\sqrt{6}}{22}\right)^9 - 9\left(\frac{9}{11} - \frac{\sqrt{6}}{22}\right)^{10}\right]
          \end{aligned} \\
          &= 1 - \frac{29069887257\sqrt{6}}{103749698404} \\
          &\approx .3137
        \end{align*}
    \end{enumerate}
  \item[21.]
    An ecologist wishes to mark off a circular sampling region having radius 10 m. However, the radius of the resulting region is actually a random variable $R$ with pdf
    \begin{align*}
      f(r) = \begin{cases}
        \frac{3}{4}[1 - (10 - r)^2] & 9 \le r \le 11 \\
        0                           & \text{otherwise}
      \end{cases}
    \end{align*}
    What is the expected area of the resulting circular region?
\end{enumerate}

\end{document}
