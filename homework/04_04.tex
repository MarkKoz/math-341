\documentclass[letterpaper,12pt]{article}
\usepackage[section=4.4]{mathhw}
\usepackage{physics}

\begin{document}

\maketitle

\begin{enumerate}
  \item[59.]
    Let $X =$ the time between two successive arrivals at the drive-up window of a local bank. If $X$ has an exponential distribution with $\lambda = 1$ (which is identical to a standard gamma distribution with $\alpha = 1$), compute the following:
    \begin{enumerate}
      \item[a.]
        The expected time between two successive arrivals
        \begin{align*}
          \mu &= \frac{1}{\lambda} = 1
        \end{align*}
      \item[b.]
        The standard deviation of the time between successive arrivals
        \begin{align*}
          \sigma &= \frac{1}{\lambda} = 1
        \end{align*}
      \item[c.]
        $P(X \le 4)$
        \begin{align*}
          P(X \le 4) &= F(4; 1) \\
          &= 1 - e^{-1 \times 4} \\
          &= 1 - e^{-4} \\
          &\approx .981684
        \end{align*}
      \item[d.]
        $P(2 \le X \le 5)$
        \begin{align*}
          P(2 \le X \le 5) &= F(5; 1) - F(2; 1) \\
          &= (1 - e^{-1 \times 5}) - (1 - e^{-1 \times 2}) \\
          &= -e^{-5} + e^{-2} \\
          &\approx .128597
        \end{align*}
    \end{enumerate}
  \item[60.]
    Let $X$ denote the distance (m) that an animal moves from its birth site to the first territorial vacancy it encounters. Suppose that for banner-tailed kangaroo rats, $X$ has an exponential distribution with parameter $\lambda = .01386$ (as suggested in the article ``Competition and Dispersal from Multiple Nests,'' \textit{Ecology}, 1997: 873–883).
    \begin{enumerate}
      \item[a.]
        What is the probability that the distance is at most 100 m? At most 200 m? Between 100 and 200 m?
        \begin{align*}
          P(X \le 100) &= F(100; .01386) \\
          &= 1 - e^{-.01386 \times 100} \\
          &= 1 - e^{-1.386} \\
          &\approx .749926
        \end{align*}
        \begin{align*}
          P(X \le 200) &= F(200; .01386) \\
          &= 1 - e^{-.01386 \times 200} \\
          &= 1 - e^{-2.772} \\
          &\approx .937463
        \end{align*}
        \begin{align*}
          P(100 \le X \le 200) &= F(200; .01386) - F(100; .01386) \\
          &\approx .937463 - .749926 \\
          &\approx .187537
        \end{align*}
      \item[b.]
        What is the probability that distance exceeds the mean distance by more than 2 standard deviations?
        \begin{align*}
          \mu &= \sigma = \frac{1}{\lambda} = \frac{1}{.01386}
        \end{align*}
        \begin{align*}
          P(|X| > \mu + 2\sigma) &= 1 - P(|X| \le \mu + 2\sigma) \\
          &= 1 - P(X \le 3\mu) \\
          &= 1 - P\left(X \le \frac{1}{.01386} + \frac{2}{.01386}\right) \\
          &= 1 - P\left(X \le \frac{3}{.01386}\right) \\
          &= 1 - F\left(\frac{3}{.01386}; .01386\right) \\
          &= 1 - (1 - e^{\frac{-.01386 \times 3}{.01386}}) \\
          &= 1 - (1 - e^{-3}) \\
          &= e^{-3} \\
          &\approx .0497871
        \end{align*}
        $P(X \le \mu - 2\sigma)$ is discarded since $\mu > 0$, $F(x; \lambda) = 0$ for $x < 0$, and
        \begin{align*}
          x = \mu - 2\sigma = \mu - 2\mu = -\mu = < 0
        \end{align*}
      \item[c.]
        What is the value of the median distance?
        \begin{align*}
          F(x; .01386) &= \frac{1}{2} \\
          1 - e^{-.01386x} &= \frac{1}{2} \\
          \frac{1}{e^{.01386x}} &= \frac{1}{2} \\
          e^{.01386x} &= 2 \\
          .01386x &= ln(2) \\
          x &= \frac{ln(2)}{.01386} \\
          x &\approx 50.0106
        \end{align*}
    \end{enumerate}
  \item[65.]
    Let $X$ denote the data transfer time (ms) in a grid computing system (the time required for data transfer between a ``worker'' computer and a ``master'' computer. Suppose that $X$ has a gamma distribution with mean value 37.5 ms and standard deviation 21.6 (suggested by the article ``Computation Time of Grid Computing with Data Transfer Times that Follow a Gamma Distribution,'' \textit{Proceedings of the First International Conference on Semantics, Knowledge, and Grid}, 2005).
    \begin{enumerate}
      \item[a.]
        What are the values of $\alpha$ and $\beta$?
        \begin{align*}
          \mu &= \alpha\beta = 37.5 \\
          \sigma^2 &= \alpha\beta^2 = 21.6^2 = 466.56
        \end{align*}
        \begin{align*}
          \alpha\beta^2 &= 466.56 \\
          \frac{37.5}{\beta} \times \beta^2 &= 466.56 \\
          37.5\beta &= 466.56 \\
          \beta &= \frac{466.56}{37.5} \\
          \beta &\approx 12.44
        \end{align*}
        \begin{align*}
          \alpha &= \frac{37.5}{\beta} = \frac{37.5}{\frac{466.56}{37.5}} = \frac{1406.25}{466.56} \approx 3
        \end{align*}
      \item[b.]
        What is the probability that data transfer time exceeds 50 ms?
        \begin{align*}
          P(X > 50) &= 1 - P(X \le 50) \\
          &\approx 1 - F(50; 3, 12.44) \\
          &\approx 1 - F\left(\frac{50}{12.44}; 3\right) \\
          &\approx 1 - F(4; 3) \\
          &\approx 1 - .762 \\
          &\approx .238
        \end{align*}
      \item[c.]
        What is the probability that data transfer time is between 50 and 75 ms?
        \begin{align*}
          P(50 \le X \le 75) &= F(75; 3, 12.44) - F(50; 3, 12.44) \\
          &\approx F\left(\frac{75}{12.44}; 3\right) - .762 \\
          &\approx F(6; 3) - .762 \\
          &\approx .938 - .762 \\
          &\approx .176
        \end{align*}
    \end{enumerate}
  \item[67.]
    Suppose that when a transistor of a certain type is subjected to an accelerated life test, the lifetime $X$ (in weeks) has a gamma distribution with mean 24 weeks and standard deviation 12 weeks.
    \begin{align*}
          \mu &= \alpha\beta = 24 \\
          \sigma^2 &= \alpha\beta^2 = 12^2 = 144
        \end{align*}
        \begin{align*}
          \alpha\beta^2 &= 144 \\
          \frac{24}{\beta} \times \beta^2 &= 144 \\
          24\beta &= 144 \\
          \beta &= \frac{144}{24} \\
          \beta &= 6
        \end{align*}
        \begin{align*}
          \alpha &= \frac{24}{\beta} = \frac{24}{6} = 4
        \end{align*}
    \begin{enumerate}
      \item[a.]
        What is the probability that a transistor will last between 12 and 24 weeks?
        \begin{align*}
          P(12 \le X \le 24) &= F(24; 4, 6) - F(12; 4, 6) \\
          &= F\left(\frac{24}{6}; 4\right) - F\left(\frac{12}{6}; 4\right) \\
          &= F(4; 4) - F(2; 4) \\
          &\approx .567 - .143 \\
          &\approx .424
        \end{align*}
      \item[b.]
        What is the probability that a transistor will last at most 24 weeks? Is the median of the lifetime distribution less than 24? Why or why not?
        \begin{align*}
          P(X \le 24) &= F(24; 4, 6) \\
          &\approx .567
        \end{align*}
        The use of the \textit{cumulative} distribution function means the probability increases as $x$ increases i.e.\ $P(X \le x) \le P(X \le x + 1)$. Thus, when $P(X \le \tilde{\mu}) \le P(X \le \mu)$, $\tilde{\mu} \le \mu$.
        \begin{align*}
          P(X \le \tilde{\mu}) = .5 < .567 \approx P(X \le \mu = 24) \implies \tilde{\mu} \le 24
        \end{align*}
      \item[c.]
        What is the 99th percentile of the lifetime distribution?
        \begin{align*}
          F(\eta(.99); 4, 6) &= .99 \\
          F\left(\frac{\eta(.99)}{6}; 4\right) &= .99
        \end{align*}
        According to Appendix A.4, $F(10; 4) \approx .990$. Thus,
        \begin{align*}
          \frac{\eta(.99)}{6} &\approx 10 \\
          \eta(.99) &\approx 10 \times 6 \\
          \eta(.99) &\approx 60
        \end{align*}
      \item[d.]
        Suppose the test will actually be terminated after $t$ weeks. What value of $t$ is such that only .5\% of all transistors would still be operating at termination?
        \begin{align*}
          P(X > t) &= .005 \\
          1 - P(X \le t) &= .005 \\
          P(X \le t) &= .995 \\
          F(t; 4, 6) &= .995 \\
          F\left(\frac{t}{6}; 4\right) &= .995
        \end{align*}
        According to Appendix A.4, $F(11; 4) \approx .995$. Thus,
        \begin{align*}
          \frac{t}{6} &\approx 11 \\
          t &\approx 11 \times 6 \\
          t &\approx 66
        \end{align*}
    \end{enumerate}
\end{enumerate}

\end{document}
