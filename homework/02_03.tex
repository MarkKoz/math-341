\documentclass[letterpaper,12pt]{article}
\usepackage[section=2.3]{mathhw}
\usepackage[makeroom]{cancel}
\usepackage{pgf}

% https://tex.stackexchange.com/a/107126
\usepackage{mathtools}
\newcommand\perm[2]{\prescript{^{#1}\mkern-2.5mu}{}P_{#2}}

\newcommand\calc[1]{\pgfmathparse{#1}\pgfmathprintnumber{\pgfmathresult}}

\newcommand\comba[2]{\frac{\perm{#1}{#2}}{#2!}}
\newcommand\combb[2]{\frac{#1!}{#2!(#1 - #2)!}}
\newcommand\combc[2]{\frac{#1!}{#2! \cdot \calc{#1 - #2}!}}

\begin{document}

\maketitle

\begin{enumerate}
  \item[29.]
    As of April 2006, roughly 50 million .com web domain names were registered (e.g., yahoo.com).
    \begin{enumerate}
      \item[a.]
        How many domain names consisting of just two letters in sequence can be formed? How many domain names of length two are there if digits as well as letters are permitted as characters? [\textit{Note:} A character length of three or more is now mandated.] \\
        \begin{minipage}{.5\linewidth}
          \begin{align*}
            n &= 26 * 26 \\
            n &= 26^2 \\
            &= 676
          \end{align*}
        \end{minipage}%
        \begin{minipage}{.5\linewidth}
          \begin{align*}
            n &= (26 + 10)^2 \\
            &= 36^2 \\
            &= 1296
          \end{align*}
        \end{minipage}
      \item[b.]
        How many domain names are there consisting of three letters in sequence? How many of this length are there if either letters or digits are permitted? [\textit{Note:} All are currently taken.] \\
        \begin{minipage}{.5\linewidth}
          \begin{align*}
            n &= 26 * 26 * 26 \\
            n &= 26^3 \\
            &= 17576
          \end{align*}
        \end{minipage}%
        \begin{minipage}{.5\linewidth}
          \begin{align*}
            n &= (26 + 10)^3 \\
            &= 36^3 \\
            &= 46656
          \end{align*}
        \end{minipage}
      \item[c.]
        Answer the questions posed in (b) for four-character sequences.
        \begin{minipage}{.5\linewidth}
          \begin{align*}
            n &= 26 * 26 * 26 * 26 \\
            n &= 26^4 \\
            &= 456976
          \end{align*}
        \end{minipage}%
        \begin{minipage}{.5\linewidth}
          \begin{align*}
            n &= (26 + 10)^4 \\
            &= 36^4 \\
            &= 1679616
          \end{align*}
        \end{minipage}
      \item[d.]
        As of April 2006, 97,786 of the four-character sequences using either letters or digits had not yet been claimed. If a four-character name is randomly selected, what is the probability that it is already owned?
        \begin{align*}
          P = \frac{36^4 - 97786}{36^4} = \frac{1581830}{1679616} \approx 0.94178
        \end{align*}
    \end{enumerate}
  \item[30.]
    A friend of mine is giving a dinner party. His current wine supply includes 8 bottles of zinfandel, 10 of merlot, and 12 of cabernet (he only drinks red wine), all from different wineries.
    \begin{enumerate}
      \item[a.]
        If he wants to serve 3 bottles of zinfandel and serving order is important, how many ways are there to do this?
        \begin{align*}
          \perm{8}{3} = \frac{8!}{(8 - 3)!} = \frac{8!}{5!} = \frac{8 \cdot 7 \cdot 6 \cdot \cancel{5!}}{\cancel{5!}} = 8 \cdot 7 \cdot 6 = 336
        \end{align*}
      \item[b.]
        If 6 bottles of wine are to be randomly selected from the 30 for serving, how many ways are there to do this?
        \begin{align*}
          \binom{30}{6} = \comba{30}{6} = \combb{30}{6} = \combc{30}{6} = 593775
        \end{align*}
      \item[c.]
        If 6 bottles are randomly selected, how many ways are there to obtain two bottles of each variety?
        \begin{align*}
          \binom{8}{2} \binom{10}{2} \binom{12}{2} &= \comba{8}{2} \times \comba{10}{2} \times \comba{12}{2} \\
          &= \combb{8}{2} \times \combb{10}{2} \times \combb{12}{2} \\
          &= \combc{8}{2} \times \combc{10}{2} \times \combc{12}{2} \\
          &= 28 \cdot 45 \cdot 66 \\
          &= 83160
        \end{align*}
      \item[d.]
        If 6 bottles are randomly selected, what is the probability that this results in two bottles of each variety being chosen?
        \begin{align*}
          P = \frac{\binom{8}{2} \binom{10}{2} \binom{12}{2}}{\binom{30}{6}} = \frac{83160}{593775} \approx 0.14
        \end{align*}
      \item[e.]
        If 6 bottles are randomly selected, what is the probability that all of them are the same variety?
        \begin{align*}
          P(Z_6 \cup M_6 \cup C_6) &= P(Z_6) + P(M_6) + P(C_6) \\
          &= \frac{\binom{8}{6} + \binom{10}{6} + \binom{12}{6}}{\binom{30}{6}} \\
          % &= \frac{\comba{8}{6} + \comba{10}{6} + \comba{12}{6}}{593775} \\
          % &= \frac{\combb{8}{6} + \combb{10}{6} + \combb{12}{6}}{593775} \\
          &= \frac{\combc{8}{6} + \combc{10}{6} + \combc{12}{6}}{593775} \\
          &= \frac{28 + 210 + 924}{593775} \\
          &= \frac{1162}{593775} \\
          &\approx 0.002
        \end{align*}
    \end{enumerate}
  \item[34.]
    Computer keyboard failures can be attributed to electrical defects or mechanical defects. A repair facility currently has 25 failed keyboards, 6 of which have electrical defects and 19 of which have mechanical defects.
    \begin{enumerate}
      \item[a.]
        How many ways are there to randomly select 5 of these key boards for a thorough inspection (without regard to order)?
        \begin{align*}
          \binom{25}{5} = \comba{25}{5} = \combb{25}{5} = \combc{25}{5} = 53130
        \end{align*}
      \item[b.]
        In how many ways can a sample of 5 keyboards be selected so that exactly two have an electrical defect?
        \begin{align*}
          \binom{6}{2} \binom{19}{3} = \comba{6}{2} \times \comba{19}{3} = \combc{6}{2} \times \combc{19}{3} = 15 \cdot 969 = 14535
        \end{align*}
      \item[c.]
        If a sample of 5 keyboards is randomly selected, what is the probability that at least 4 of these will have a mechanical defect?
        \begin{align*}
          P(M_4 \cup M_5) &= P(M_4) + P(M_5) \\
          &= \frac{N_4 + N_5}{N} \\
          &= \frac{\binom{6}{1} \binom{19}{4} + \binom{6}{0} \binom{19}{5}}{\binom{25}{5}} \\
          &= \frac{(6)\comba{19}{4} + (1)\comba{19}{5}}{53130} \\
          &= \frac{(6)\combc{19}{4} + \combc{19}{5}}{53130} \\
          &= \frac{6 \cdot 3876 + 11628}{53130} \\
          &= \frac{34884}{53130} \\
          &\approx 0.657
        \end{align*}
    \end{enumerate}
  \item[35.]
    A production facility employs 10 workers on the day shift, 8 workers on the swing shift, and 6 workers on the graveyard shift. A quality control consultant is to select 5 of these workers for in-depth interviews. Suppose the selection is made in such a way that any particular group of 5 workers has the same chance of being selected as does any other group (drawing 5 slips without replacement from among 24).
    \begin{enumerate}
      \item[a.]
        How many selections result in all 5 workers coming from the day shift? What is the probability that all 5 selected workers will be from the day shift?
      \item[b.]
        What is the probability that all 5 selected workers will be from the same shift?
      \item[c.]
        What is the probability that at least two different shifts will be represented among the selected workers?
      \item[d.]
        What is the probability that at least one of the shifts will be unrepresented in the sample of workers?
    \end{enumerate}
  \item[38.]
    A sonnet is a 14-line poem in which certain rhyming patterns are followed. The writer Raymond Queneau published a book containing just 10 sonnets, each on a different page. However, these were structured such that other sonnets could be created as follows: the first line of a sonnet could come from the first line on any of the 10 pages, the second line could come from the second line on any of the 10 pages, and so on (successive lines were perforated for this purpose).
    \begin{enumerate}
        \item[a.]
          How many sonnets can be created from the 10 in the book?
        \item[b.]
          If one of the sonnets counted in part (a) is selected at random, what is the probability that none of its lines came from either the first or the last sonnet in the book?
      \end{enumerate}
\end{enumerate}

\end{document}
