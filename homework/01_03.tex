\documentclass[letterpaper,12pt]{article}
\usepackage[section=1.2]{mathhw}
\usepackage{plotting}
\usepackage{siunitx}

\begin{document}

\maketitle

\begin{enumerate}
  \item[33.]
    The May 1, 2009, issue of The Montclarian reported the following home sale amounts for a sample of homes in Alameda, CA that were sold the previous month (1000s of \$):
    \begin{center}
      \begin{tabular}{cccccccccc}
        590 & 815 & 575 & 608 & 350 & 1285 & 408 & 540 & 555 & 679
      \end{tabular}
    \end{center}
    \begin{enumerate}
      \item[a.]
        Calculate and interpret the sample mean and median.
        \begin{align*}
          \bar{x} = \frac{\sum_{i=1}^{n} x_{i}}{n} = \frac{6405}{10} = 640.5
        \end{align*}
        Sorted values:
        \begin{center}
          \begin{tabular}{cccccccccc}
            350 & 408 & 540 & 555 & 575 & 590 & 608 & 679 & 815 & 1285
          \end{tabular}
        \end{center}
        \begin{align*}
          \tilde{x} = \frac{x_{\frac{n}{2}} + x_{\frac{n}{2} + 1}}{2} = \frac{x_{5} + x_{6}}{2} = \frac{575 + 590}{2} = \frac{1165}{2} = 582.5
        \end{align*}
        The sample mean is larger than the sample median. This is evidence of how the mean is more susceptible to outliers than the median. In this sample, there is an outlier at the end, 1285. Hence, the sample mean is positively skewed.
      \item[b.]
        Suppose the $6\textsuperscript{th}$ observation had been 985 rather than 1285. How would the mean and median change? \\ \\
        The median would not change at all, because $x_{5}$ and $x_{6}$ (when sorted) would still be the same values. However, the mean would decrease. As mentioned in (a), the sample mean was positively skewed because of the outlier 1285. Thus, the outlier decreasing would result in the sample mean also decreasing.
    \end{enumerate}
  \item[36.]
    A sample of 26 offshore oil workers took part in a simulated escape exercise, resulting in the accompanying data on time (sec) to complete the escape (``Oxygen Consumption and Ventilation During Escape from an Offshore Platform,'' \textit{Ergonomics}, 1997: 281–292):
    \begin{center}
      \begin{tabular}{ccccccccc}
        389 & 356 & 359 & 363 & 375 & 424 & 325 & 394 & 402 \\
        373 & 373 & 370 & 364 & 366 & 364 & 325 & 339 & 393 \\
        392 & 369 & 374 & 359 & 356 & 403 & 334 & 397
      \end{tabular}
    \end{center}
    \begin{enumerate}
      \item[a.]
        Construct a stem-and-leaf display of the data. How does it suggest that the sample mean and median will compare?
        \begin{center}
          \begin{tabular}{c|ccccc}
            32 & 5 & 5 \\
            33 & 4 & 9 \\
            35 & 6 & 6 & 9 & 9 \\
            36 & 3 & 4 & 4 & 6 & 9 \\
            37 & 0 & 3 & 3 & 4 & 5 \\
            38 & 9 \\
            39 & 2 & 3 & 4 & 7 \\
            40 & 2 & 3 \\
            42 & 4 \\
            \multicolumn{6}{r}{Key: 32 | 5 = 325}
          \end{tabular}
        \end{center}
        The display suggests that the sample mean and sample median are quite similar. Stem 38 is sparse, but this should only negatively skew the mean very slightly. Besides that, the display looks like a symmetric bell curve, which would imply the sample mean and sample median are the same. There aren't any strong outliers on either end that would skew the mean.
      \item[b.]
        Calculate the values of the sample mean and median. [\textit{Hint:} $\sum{x_i} = 9638$.]
        \begin{align*}
          \bar{x} = \frac{\sum_{i=1}^{n} x_{i}}{n} = \frac{9638}{26} \approx 370.7
        \end{align*}
        \begin{align*}
          \tilde{x} = \frac{x_{\frac{n}{2}} + x_{\frac{n}{2} + 1}}{2} = \frac{x_{13} + x_{14}}{2} = \frac{369 + 370}{2} = \frac{739}{2} = 369.5
        \end{align*}
    \end{enumerate}
  \item[38.]
    Blood pressure values are often reported to the nearest 5 mmHg (100, 105, 110, etc.). Suppose the actual blood pressure values for nine randomly selected individuals are
    \begin{center}
      \begin{tabular}{ccccccc}
        118.6 & 127.4 & 138.4 & 130.0 & 113.7 & 122.0 & 108.3 \\
        131.5 & 133.2
      \end{tabular}
    \end{center}
    \begin{enumerate}
      \item[a.]
        What is the median of the \textit{reported} blood pressure values?
        \\ \\
        Sorted and normalised values:
        \begin{center}
          \begin{tabular}{cccccccccc}
            110 & 115 & 120 & 120 & 125 & 130 & 130 \\
            135 & 140
          \end{tabular}
        \end{center}
        \begin{align*}
          \tilde{x} = x_{\frac{n}{2}} = x_{5} = 125
        \end{align*}
      \item[b.]
        Suppose the blood pressure of the second individual is 127.6 rather than 127.4 (a small change in a single value). How does this affect the median of the reported values? What does this say about the sensitivity of the median to rounding or grouping in the data?
        \\ \\
        Because $x_{\frac{n}{2}} = 127.4$, changing this value would directly affect the median. This change would cause the value to be rounded up to 130, rather than down to 125. Thus, the median would increase. This shows that the median is sensitive to minor changes when it's at the threshold of being rounded in the other direction.
    \end{enumerate}
  \item[41.]
    A sample of $n = 10$ automobiles was selected, and each was subjected to a 5-mph crash test. Denoting a car with no visible damage by S (for success) and a car with such damage by F, results were as follows:
    \begin{center}
      \begin{tabular}{cccccccccc}
        S & S & F & S & S & S & F & F & S & S
      \end{tabular}
    \end{center}
    \begin{enumerate}
      \item[a.]
        What is the value of the sample proportion of successes $x/n$?
        \begin{align*}
          \frac{x}{n} = \frac{\text{number of successes}}{\text{total}} = \frac{7}{10} = 0.7
        \end{align*}
      \item[b.]
        Replace each S with a 1 and each F with a 0. Then calculate $\bar{x}$ for this numerically coded sample. How does $\bar{x}$ compare to $x/n$?
        \begin{align*}
          \bar{x} = \frac{1 + 1 + 0 + 1 + 1 + 1 + 0 + 0 + 1 + 1}{10} = \frac{7}{10} = 0.7
        \end{align*}
        Thus, $\bar{x}$ is equivalent to $x/n$.
      \item[c.]
        Suppose it is decided to include 15 more cars in the experiment. How many of these would have to be S's to give $x/n = .80$ for the entire sample of 25 cars?
        \begin{align*}
          \frac{x}{n} = \frac{x}{25} = 0.8 \\
          x = 0.8 \cdot 25 = 20
        \end{align*}
        Thus, there would need to be 20 S's in total to give $x/n = .80$. Because the original sample of 10 already had 7 S's, only $20 - 7 = 13$ more S's would be needed. Thus, 13 out of the 15 new cars added would need to be S's.
    \end{enumerate}
\end{enumerate}

\end{document}
