\documentclass[letterpaper,12pt]{article}
\usepackage[section=1.2]{mathhw}
\usepackage{plotting}
\usepackage{siunitx}

\begin{document}

\maketitle

\begin{enumerate}
  \item[33.]
    The May 1, 2009, issue of The Montclarian reported the following home sale amounts for a sample of homes in Alameda, CA that were sold the previous month (1000s of \$):
    \begin{center}
      \begin{tabular}{cccccccccc}
        590 & 815 & 575 & 608 & 350 & 1285 & 408 & 540 & 555 & 679
      \end{tabular}
    \end{center}
    \begin{enumerate}
      \item[a.]
        Calculate and interpret the sample mean and median.
      \item[b.]
        Suppose the $6\textsuperscript{th}$ observation had been 985 rather than 1285. How would the mean and median change?
    \end{enumerate}
  \item[36.]
    A sample of 26 offshore oil workers took part in a simulated escape exercise, resulting in the accompanying data on time (sec) to complete the escape (``Oxygen Consumption and Ventilation During Escape from an Offshore Platform,'' \textit{Ergonomics}, 1997: 281–292):
    \begin{center}
      \begin{tabular}{ccccccccc}
        389 & 356 & 359 & 363 & 375 & 424 & 325 & 394 & 402 \\
        373 & 373 & 370 & 364 & 366 & 364 & 325 & 339 & 393 \\
        392 & 369 & 374 & 359 & 356 & 403 & 334 & 397
      \end{tabular}
    \end{center}
    \begin{enumerate}
      \item[a.]
        Construct a stem-and-leaf display of the data. How does it suggest that the sample mean and median will compare?
      \item[b.]
        Calculate the values of the sample mean and median. [\textit{Hint:} $\sum{x_i} = 9638$.]
    \end{enumerate}
  \item[38.]
    Blood pressure values are often reported to the nearest 5 mmHg (100, 105, 110, etc.). Suppose the actual blood pressure values for nine randomly selected individuals are
    \begin{center}
      \begin{tabular}{ccccccc}
        118.6 & 127.4 & 138.4 & 130.0 & 113.7 & 122.0 & 108.3 \\
        131.5 & 133.2
      \end{tabular}
    \end{center}
    \begin{enumerate}
      \item[a.]
        What is the median of the \textit{reported} blood pressure values?
      \item[b.]
        Suppose the blood pressure of the second individual is 127.6 rather than 127.4 (a small change in a single value). How does this affect the median of the reported values? What does this say about the sensitivity of the median to rounding or grouping in the data?
    \end{enumerate}
  \item[41.]
    A sample of $n = 10$ automobiles was selected, and each was subjected to a 5-mph crash test. Denoting a car with no visible damage by S (for success) and a car with such damage by F, results were as follows:
    \begin{center}
      \begin{tabular}{cccccccccc}
        S & S & F & S & S & S & F & F & S & S
      \end{tabular}
    \end{center}
    \begin{enumerate}
      \item[a.]
        What is the value of the sample proportion of successes $x/n$?
      \item[b.]
        Replace each S with a 1 and each F with a 0. Then calculate $\bar{x}$ for this numerically coded sample. How does $\bar{x}$ compare to $x/n$?
      \item[c.]
        Suppose it is decided to include 15 more cars in the experiment. How many of these would have to be S's to give $x/n = .80$ for the entire sample of 25 cars?
    \end{enumerate}
\end{enumerate}

\end{document}
