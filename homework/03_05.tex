\documentclass[letterpaper,12pt]{article}
\usepackage[section=3.5]{mathhw}
\usepackage{plotting}

\newcommand{\hg}[4]{%
  \frac%
    {\binom{#3}{#1} \binom{#4 - #3}{#2 - #1}}%
    {\binom{#4}{#2}}%
}

\newcommand{\hgeq}[4]{%
  h(#1; #2, #3, #4) = \hg{#1}{#2}{#3}{#4}%
}

\newcommand{\nb}[3]{%
  \binom{#1 + #2 - 1}{#2 - 1}{#3}^{#2}(1 - #3)^{#1}%
}

\begin{document}

\maketitle

\begin{enumerate}
  \item[68.]
    Eighteen individuals are scheduled to take a driving test at a particular DMV office on a certain day, eight of whom will be taking the test for the first time. Suppose that six of these individuals are randomly assigned to a particular examiner, and let X be the number among the six who are taking the test for the first time.
    \begin{enumerate}
      \item[a.]
        What kind of a distribution does $X$ have (name and values of all parameters)?
        \\ \\
        Hypergeometric distribution where $n = 6$, $M = 8$, and $N = 18$.
      \item[b.]
        Compute $P(X = 2)$, $P(X \le 2)$, and $P(X \ge 2)$.
        \begin{align*}
          P(X = 2) &= \hgeq{2}{6}{8}{18} = \frac{28 \times 210}{18564} \approx .317 \\
          P(X \le 2) &= \sum_{x = 0}^{2} h(x; 6, 8, 18) \\
          &= \sum_{x = 0}^{2} \hg{x}{6}{8}{18} \\
          &= \frac{5}{442} + \frac{24}{221} + \frac{70}{221} \\
          &\approx .437 \\
          P(X \ge 2) &= 1 - P(X \le 2) + P(X = 2) \\
          &\approx 1 - .437 + .317 \\
          &\approx .880
        \end{align*}
      \item[c.]
        Calculate the mean value and standard deviation of $X$.
        \begin{align*}
          p &= \frac{M}{N} = \frac{8}{18} \\
          E(X) &= np = 6 \cdot \frac{8}{18} = \frac{8}{3} = 2.\bar{6} \\
          \sigma &= \sqrt{\frac{N - n}{N - 1} \cdot E(X) \cdot (1 - p)} \\
          &= \sqrt{\frac{18 - 6}{18 - 1} \cdot \frac{8}{3} \cdot \left(1 - \frac{8}{18}\right)} \\
          &= \sqrt{\frac{160}{153}} \\
          &\approx 1.023
        \end{align*}
    \end{enumerate}
  \item[72.]
    A personnel director interviewing 11 senior engineers for four job openings has scheduled six interviews for the first day and five for the second day of interviewing. Assume that the candidates are interviewed in random order.
    \begin{enumerate}
      \item[a.]
        What is the probability that x of the top four candidates are interviewed on the first day?
        \begin{align*}
          P(X = x) &= \hgeq{x}{6}{4}{11}
        \end{align*}
      \item[b.]
        How many of the top four candidates can be expected to be interviewed on the first day?
        \begin{align*}
          E(X) &= n \cdot \frac{M}{N} = 6 \cdot \frac{4}{11} = \frac{24}{11} \approx 2
        \end{align*}
    \end{enumerate}
  \item[75.]
    The probability that a randomly selected box of a certain type of cereal has a particular prize is .2. Suppose you purchase box after box until you have obtained two of these prizes.
    \begin{enumerate}
      \item[a.]
        What is the probability that you purchase $x$ boxes that do not have the desired prize?
        \begin{align*}
          P(X = x) &= nb(x; 2, .2) = \nb{x}{2}{.2}
        \end{align*}
      \item[b.]
        What is the probability that you purchase four boxes?
        \begin{align*}
          P(X = 2) &= nb(2; 2, .2) \\
          &= \nb{2}{2}{.2} \\
          &= \binom{3}{1}.2^2.8^2 \\
          &= .0768
        \end{align*}
        Because $r = 2$, $X = 4 - r = 2$.
      \item[c.]
        What is the probability that you purchase at most four boxes?
        \begin{align*}
          P(X \le 2) &= \sum_{x = 0}^{2} nb(x; 2, .2) \\
          &= \sum_{x = 0}^{2} \nb{x}{2}{.2} \\
          &= .04 + .064 + .0768 + \\
          &= .1808
        \end{align*}
      \item[d.]
        How many boxes without the desired prize do you expect to purchase?
        \begin{align*}
          E(X) &= \frac{r(1 - p)}{p} = \frac{2(1 - .2)}{.2} = 8
        \end{align*}
        How many boxes do you expect to purchase?
        \begin{align*}
          E(X) + r &= 8 + 2 = 10
        \end{align*}
    \end{enumerate}
  \item[76.]
    A family decides to have children until it has three children of the same gender. Assuming $P(B) = P(G) = .5$, what is the pmf of $X =$ the number of children in the family?
    \\ \\
    First, notice the only possible number of children is 3, 4, or 5. $X < 3$ is impossible because the condition to stop requires that there are \textit{three} children of the same gender. $X > 5$ is impossible because at $X = 5$ it is guaranteed that there are 3 children of the same gender, so there'd be no reason to continue.
    \begin{align*}
      P(X = 3) &= P(BBB) + P(GGG) = 2 \times .5^3 = .25 \\
      P(X = 4) &= \begin{aligned}[t]
        &P(\text{2 of first 3 are B and 4th is B}) \\
        + &P(\text{2 of first 3 are G and 4th is G})
      \end{aligned} \\
      &= 2 \times \binom{3}{2} \times .5^4 \\
      &= .375 \\
      P(X = 5) &= 1 - P(X = 4) - P(X = 3) = 1 - .375 - .25 = .375
    \end{align*}
\end{enumerate}

\end{document}
