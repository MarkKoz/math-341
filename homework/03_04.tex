\documentclass[letterpaper,12pt]{article}
\usepackage[section=3.4]{mathhw}

\newcommand{\bp}[3]{%
  \binom{#2}{#1}#3^{#1}(1 - #3)^{#2 - #1}%
}

\begin{document}

\maketitle

\begin{enumerate}
  \item[46.]
    Compute the following binomial probabilities directly from the formula for $b(x; n, p)$:
    \begin{enumerate}
      \item[a.]
        $b(3; 8, .35)$
        \begin{align*}
          b(3; 8, .35) &= \bp{3}{8}{.35} \\
          &\approx 56 \times .042875 \times .1160290625 \\
          &\approx .2785857790625
        \end{align*}
      \item[b.]
        $b(5; 8, .6)$
        \begin{align*}
          b(5; 8, .6) &= \bp{5}{8}{.6}  \\
          &= 56 \times .07776 \times 0.064 \\
          &= .27869184
        \end{align*}
      \item[c.]
        $P(3 \le X \le 5)$ when $n = 7$ and $p = .6$
        \begin{align*}
          P(3 \le X \le 5) &= \sum_{y = 3}^5 b(y; 7; .6) \\
          &= \sum_{y = 3}^5 \bp{y}{7}{.6}  \\
          &= .193536 + .290304 + .2612736 \\
          &= .7451136
        \end{align*}
      \item[d.]
        $P(1 \le X)$ when $n = 9$ and $p = .1$
        \begin{align*}
          P(1 \le X) &= 1 - P(X = 0) \\
          &= 1 - b(0; 9; .1) \\
          &= 1 - \bp{0}{9}{.1} \\
          &\approx 1 - 0.38742 \\
          &\approx 0.61258
        \end{align*}
    \end{enumerate}
  \item[48.]
    NBC News reported on May 2, 2013, that 1 in 20 children in the United States have a food allergy of some sort. Consider selecting a random sample of 25 children and let $X$ be the number in the sample who have a food allergy. Then $X \sim$ Bin$(25, .05)$.
    \begin{enumerate}
      \item[a.]
        Determine both $P(X \le 3)$ and $P(X < 3)$.
        \begin{align*}
          P(X < 3) &= \sum_{y = 0}^3 b(y; 25; .05) \\
          &= \sum_{y = 0}^2 \bp{y}{25}{.05} \\
          &\approx .27739 + .364986 + .230518 \\
          &\approx .872894 \\
          \\
          P(X \le 3) &= P(X < 3) + b(3; 25; .05) \\
          &\approx .872894 + \bp{3}{25}{.05} \\
          &\approx .872894 + .0930159 \\
          &\approx .9659099
        \end{align*}
      \item[b.]
        Determine $P(X \ge 4)$.
        \begin{align*}
          P(X \ge 4) &= 1 - P(X \le 3) \\
          &\approx 1 - .9659099 \\
          &\approx .0340901
        \end{align*}
      \item[c.]
        Determine $P(1 \le X \le 3)$.
        \begin{align*}
          P(1 \le X \le 3) &= P(X \le 3) - P(X = 0) \\
          &\approx .9659099 - .27739 \\
          &\approx .6885199
        \end{align*}
      \item[d.]
        What are $E(X)$ and $\sigma_x$?
        \begin{align*}
          E(X) &= np = 25 \times .05 = 1.25 \\
          \sigma_x &= \sqrt{np(1 - p)} = \sqrt{1.25(1 - .05)} = \sqrt{1.1875} \approx 1.08972
        \end{align*}
      \item[e.]
        In a sample of 50 children, what is the probability that none has a food allergy?
        \begin{align*}
          P(X = 0) &= b(0; 50; .05) \\\
          &= \bp{0}{50}{.05} \\
          &\approx 1 \times 1 \times .076945 \\
          &\approx .076945
        \end{align*}
    \end{enumerate}
  \item[50.]
    A particular telephone number is used to receive both voice calls and fax messages. Suppose that 25\% of the incoming calls involve fax messages, and consider a sample of 25 incoming calls. What is the probability that
    \begin{enumerate}
      \item[a.]
        At most 6 of the calls involve a fax message?
        \begin{align*}
          P(X \le 6) &= \sum_{y = 0}^6 b(y; 25; .25) \\
          &= \sum_{y = 0}^6 \bp{y}{25}{.25} \\
          &\approx \begin{aligned}[t]
              &.000753 + .006271 + .025085 + .064106 \\
            + &.117527 + .164538 + .182820
          \end{aligned} \\
          &\approx .561098
        \end{align*}
      \item[b.]
        Exactly 6 of the calls involve a fax message?
        \begin{align*}
          P(X = 6) &= b(6; 25; .25) \\
          &= \bp{6}{25}{.25} \\
          &\approx .182820
        \end{align*}
      \item[c.]
        At least 6 of the calls involve a fax message?
        \begin{align*}
          P(X \ge 6) &= 1 - P(x < 6) \\
          &= 1 - P(x \le 6) + P(X = 6) \\
          &\approx 1 - .561098 + .182820 \\
          &\approx .621722
        \end{align*}
      \item[d.]
        More than 6 of the calls involve a fax message?
        \begin{align*}
          P(X > 6) &= 1 - P(x \le 6) \\
          &\approx 1 - .561098 \\
          &\approx  .438902
        \end{align*}
    \end{enumerate}
  \item[54.]
    A particular type of tennis racket comes in a midsize version and an oversize version. Sixty percent of all customers at a certain store want the oversize version.
    \begin{enumerate}
      \item[a.]
        Among ten randomly selected customers who want this type of racket, what is the probability that at least six want the oversize version?
        \begin{align*}
          P(X \ge 6) &= \sum_{y = 6}^{10} b(y; 10; .6) \\
          &= \sum_{y = 6}^{10} \bp{y}{10}{.6} \\
          &\approx .250823 + .214991 + .120932 + .040311 + .006047 \\
          &\approx .633103
        \end{align*}
      \item[b.]
        Among ten randomly selected customers, what is the probability that the number who want the oversize version is within 1 standard deviation of the mean value?
        \begin{align*}
          \bar{x} &= np = 10 \times .6 = 6 \\
          \sigma &= \sqrt{np(1 - p)} = \sqrt{6(1 - .6)} \approx 1.549193
        \end{align*}
        \begin{align*}
          \bar{x} \pm \sigma &= [\lceil \bar{x} - \sigma \rceil, \lfloor \bar{x} + \sigma \rfloor] \\
          &\approx [\lceil 6 - 1.549193 \rceil, \lfloor 6 + 1.549193 \rfloor] \\
          &\approx [\lceil 4.450807 \rceil, \lfloor 7.549193 \rfloor] \\
          &= [5, 7]
        \end{align*}
        \begin{align*}
          P(5 \le X \le 7) &= \sum_{y = 5}^7 b(y; 10; .6) \\
          &= \sum_{y = 5}^7 \bp{y}{10}{.6} \\
          &\approx .200658 + .250823 + .214991 \\
          &\approx .666472
        \end{align*}
      \item[c.]
        The store currently has seven rackets of each version. What is the probability that all of the next ten customers who want this racket can get the version they want from current stock?
        \begin{align*}
          P(3 \le X \le 7) &= \sum_{y = 3}^7 b(y; 10; .6) \\
          &= \sum_{y = 3}^7 \bp{y}{10}{.6} \\
          &\approx .042467 + .111477 + .200658 + .250823 + .214991 \\
          &\approx .820416
        \end{align*}
    \end{enumerate}
  \item[56.]
    The College Board reports that 2\% of the 2 million high school students who take the SAT each year receive special accommodations because of documented disabilities (\textit{Los Angeles Times}, July 16, 2002). Consider a random sample of 25 students who have recently taken the test.
    \begin{enumerate}
      \item[a.]
        What is the probability that exactly 1 received a special accommodation?
      \item[b.]
        What is the probability that at least 1 received a special accommodation?
      \item[c.]
        What is the probability that at least 2 received a special accommodation?
      \item[d.]
        What is the probability that the number among the 25 who received a special accommodation is within 2 standard deviations of the number you would expect to be accommodated?
      \item[e.]
        Suppose that a student who does not receive a special accommodation is allowed 3 hours for the exam, whereas an accommodated student is allowed 4.5 hours. What would you expect the average time allowed the 25 selected students to be?
    \end{enumerate}
\end{enumerate}

\end{document}
