\documentclass[letterpaper,12pt]{article}
\usepackage[section=4.3]{mathhw}
\usepackage{plotting}

\usepackage{siunitx}
\sisetup{per-mode = symbol}

\pgfmathdeclarefunction{gauss}{2}{%
  \pgfmathparse{1 / (#2 * sqrt(2 * pi)) * exp(-((x - #1)^2) / (2 * #2^2))}%
}

\begin{document}

\maketitle

\begin{enumerate}
  \item[28.]
    Let $Z$ be a standard normal random variable and calculate the following probabilities, drawing pictures wherever appropriate. \\
    \\
    The values for $\Phi(z)$ are obtained from Table A.3 in the textbook.
    \begin{enumerate}
      \item[a.]
        $P(0 \le Z \le 2.17)$
        \begin{align*}
          P(0 \le Z \le 2.17) &= \Phi(2.17) - \Phi(0) \\
          &\approx .9850 - .5000 \\
          &\approx .4850
        \end{align*}
        \begin{center}
          \stdnormdist{0}{2.17}{}
        \end{center}
      \item[b.]
        $P(0 \le Z \le 1)$
        \begin{align*}
          P(0 \le Z \le 1) &= \Phi(1) - \Phi(0) \\
          &\approx .8413 - .5000 \\
          &\approx .3413
        \end{align*}
        \begin{center}
          \stdnormdist{0}{1}{}
        \end{center}
      \item[c.]
        $P(-2.50 \le Z \le 0)$
        \begin{align*}
          P(-2.50 \le Z \le 0) &= \Phi(0) - \Phi(-2.50) \\
          &\approx .5000 - .0062 \\
          &\approx .4938
        \end{align*}
        \begin{center}
          \stdnormdist{0}{2.50}{}
        \end{center}
      \item[d.]
        $P(-2.50 \le Z \le 2.50)$
        \begin{align*}
          P(-2.50 \le Z \le 2.50) &= \Phi(2.50) - \Phi(-2.50) \\
          &\approx .9938 - .0062 \\
          &\approx .9876
        \end{align*}
        \begin{center}
          \stdnormdist{-2.50}{2.50}{}
        \end{center}
      \item[e.]
        $P(Z \le 1.37)$
        \begin{align*}
          P(Z \le 1.37) = \Phi(1.37) \approx .9147
        \end{align*}
        \begin{center}
          \stdnormdist{-3.50}{1.37}{xtick = {1.37}}
        \end{center}
      \item[f.]
        $P(-1.75 \le Z)$
        \begin{align*}
          P(-1.75 \le Z) &= 1 - P(Z < -1.75) \\
          &= 1 - \Phi(-1.75) \\
          &\approx 1 - .0401 \\
          &\approx .9599
        \end{align*}
        \begin{center}
          \stdnormdist{-1.75}{3.50}{xtick = {-1.75}}
        \end{center}
      \item[g.]
        $P(-1.50 \le Z \le 2.00)$
        \begin{align*}
          P(-1.50 \le Z \le 2.00) &= \Phi(2.00) - \Phi(-1.50) \\
          &\approx .9772 - .0668 \\
          &\approx .9104
        \end{align*}
        \begin{center}
          \stdnormdist{-1.50}{2.00}{}
        \end{center}
      \item[h.]
        $P(1.37 \le Z \le 2.50)$
        \begin{align*}
          P(1.37 \le Z \le 2.50) &= \Phi(2.50) - \Phi(1.37) \\
          &\approx .9938 - .9147 \\
          &\approx .0791
        \end{align*}
        \begin{center}
          \stdnormdist{1.37}{2.50}{}
        \end{center}
      \item[i.]
        $P(1.50 \le Z)$
        \begin{align*}
          P(1.50 \le Z) &= 1 - P(Z < 1.50) \\
          &= 1 - \Phi(1.50) \\
          &\approx 1 - .9332 \\
          &\approx .0668
        \end{align*}
        \begin{center}
          \stdnormdist{1.50}{3.50}{xtick = {1.50}}
        \end{center}
      \item[j.]
        $P(|Z| \le 2.50)$
        \begin{align*}
          P(|Z| \le 2.50) &= P(-2.50 \le Z \le 2.50) \\
          &= \Phi(2.50) - \Phi(-2.50) \\
          &\approx .9938 - .0062 \\
          &\approx .9876
        \end{align*}
        \begin{center}
          \stdnormdist{-2.50}{2.50}{}
        \end{center}
    \end{enumerate}
  \item[29.]
    In each case, determine the value of the constant $c$ that makes the probability statement correct.
    \begin{enumerate}
      \item[a.]
        $F(c) = .9838$ \\
        \\
        According to Appendix A.3, $c \approx 2.14$.
      \item[b.]
        $P(0 \le Z \le c) = .291$
        \begin{align*}
          P(0 \le Z \le c) &= .291 \\
          \Phi(c) - \Phi(0) &= .291 \\
          \Phi(c) - .5000 &= .291 \\
          \Phi(c) &= .291 + .5000 \\
          \Phi(c) &= .791
        \end{align*}
        According to Appendix A.3, $c \approx .81$.
      \item[c.]
        $P(c \le Z) = .121$
        \begin{align*}
          P(c \le Z) &= .121 \\
          1 - P(Z \le c) &= .121 \\
          1 - \Phi(c) &= .121 \\
          \Phi(c) &= 1 - .121 \\
          \Phi(c) &= .879
        \end{align*}
        According to Appendix A.3, $c \approx 1.17$.
      \item[d.]
        $P(-c \le Z \le c) = .668$
        \begin{align*}
          P(-c \le Z \le c) &= .668 \\
          P(Z \le c) - P(Z \le -c) &= .668 \\
          P(Z \le c) - P(Z > c) &= .668 \\
          P(Z \le c) - [1 - P(Z \le c)] &= .668 \\
          2P(Z \le c) - 1 &= .668 \\
          \Phi(c) &= \frac{.668 + 1}{2} \\
          \Phi(c) &= .8340
        \end{align*}
        According to Appendix A.3, $c \approx .97$.
      \item[e.]
        $P(c \le |Z|) = .016$
        \begin{align*}
          P(c \le |Z|) &= .016 \\
          1 - P(|Z| \le c) &= .016 \\
          P(|Z| \le c) &= 1 - .016 \\
          P(-c \le Z \le c) &= .984 \\
          P(Z \le c) - P(Z > c) &= .984 \\
          P(Z \le c) - [1 - P(Z \le c)] &= .984 \\
          2P(Z \le c) - 1 &= .984 \\
          \Phi(c) &= \frac{.984 + 1}{2} \\
          \Phi(c) &= .992
        \end{align*}
        According to Appendix A.3, $c \approx 2.41$.
    \end{enumerate}
  \item[30.]
    Find the following percentiles for the standard normal distribution. Interpolate where appropriate.
    \begin{enumerate}
      \item[a.]
        91st
        \begin{align*}
          P(Z \le c) &= .91
        \end{align*}
        According to Appendix A.3, $c \approx 1.34$.
      \item[b.]
        9th
        \begin{align*}
          P(Z \le c) &= .09
        \end{align*}
        According to Appendix A.3, $c \approx -1.34$.
      \item[c.]
        75th
        \begin{align*}
          P(Z \le c) &= .75
        \end{align*}
        According to Appendix A.3, $c \approx .675$.
      \item[d.]
        25th
        \begin{align*}
          P(Z \le c) &= .25
        \end{align*}
        According to Appendix A.3, $c \approx -.675$.
      \item[e.]
        6th
        \begin{align*}
          P(Z \le c) &= .06
        \end{align*}
        According to Appendix A.3, $c \approx -1.55$.
    \end{enumerate}
  \item[31.]
    Determine $z_\alpha$ for the following values of $\alpha$:
    \begin{enumerate}
      \item[a.]
        $\alpha = .0055$
        \begin{align*}
          P(Z \le z_\alpha) &= 1 - \alpha \\
          P(Z \le z_{.0055}) &= 1 - .0055 \\
          &= .9945
        \end{align*}
        According to Appendix A.3, $z_{.0055} \approx 2.54$.
      \item[b.]
        $\alpha = .09$
        \begin{align*}
          P(Z \le z_\alpha) &= 1 - \alpha \\
          P(Z \le z_{.09}) &= 1 - .09 \\
          &= .9100
        \end{align*}
        According to Appendix A.3, $z_{.09} \approx 1.34$.
      \item[c.]
        $\alpha = .663$
        \begin{align*}
          P(Z \le z_\alpha) &= 1 - \alpha \\
          P(Z \le z_{.663}) &= 1 - .663 \\
          &= .3370
        \end{align*}
        According to Appendix A.3, $z_{.09} \approx -.42$.
    \end{enumerate}
  \item[32.]
    Suppose the force acting on a column that helps to support a building is a normally distributed random variable $X$ with mean value 15.0 kips and standard deviation 1.25 kips. Compute the following probabilities by standardizing and then using Table A.3.
    \begin{enumerate}
      \item[a.]
        $P(X \le 15)$
      \item[b.]
        $P(X \le 17.5)$
      \item[c.]
        $P(X \ge 10)$
      \item[d.]
        $P(14 \le X \le 18)$
      \item[e.]
        $P(|X - 15| \le 3)$
    \end{enumerate}
  \item[36.]
    Spray drift is a constant concern for pesticide applicators and agricultural producers. The inverse relationship between droplet size and drift potential is well known. The paper ``Effects of 2,4-D Formulation and Quinclorac on Spray Droplet Size and Deposition'' (\textit{Weed Technology}, 2005: 1030–1036) investigated the effects of herbicide formulation on spray atomization. A figure in the paper suggested the normal distribution with mean \qty{1050}{\micro\meter} and standard deviation \qty{150}{\micro\meter} was a reasonable model for droplet size for water (the ``control treatment'') sprayed through a \qty{760}{\milli\liter\per\minute} nozzle.
    \begin{enumerate}
      \item[a.]
        What is the probability that the size of a single droplet is less than \qty{1500}{\micro\meter}? At least \qty{1000}{\micro\meter}?
      \item[b.]
        What is the probability that the size of a single droplet is between 1000 and \qty{1500}{\micro\meter}?
      \item[c.]
        How would you characterize the smallest \qty{2}{\percent} of all droplets?
      \item[d.]
        If the sizes of five independently selected droplets are measured, what is the probability that exactly two of them exceed \qty{1500}{\micro\meter}?
    \end{enumerate}
\end{enumerate}

\end{document}
