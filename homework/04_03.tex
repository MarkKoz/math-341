\documentclass[letterpaper,12pt]{article}
\usepackage[section=4.3]{mathhw}

\usepackage{siunitx}
\sisetup{per-mode = symbol}

\begin{document}

\maketitle

\begin{enumerate}
  \item[28.]
    Let $Z$ be a standard normal random variable and calculate the following probabilities, drawing pictures wherever appropriate. \\
    \\
    The values for $\Phi(z)$ are obtained from Table A.3 in the textbook.
    \begin{enumerate}
      \item[a.]
        $P(0 \le Z \le 2.17)$
        \begin{align*}
          P(0 \le Z \le 2.17) &= \Phi(2.17) - \Phi(0) \\
          &\approx .9850 - .5000 \\
          &\approx .4850
        \end{align*}
      \item[b.]
        $P(0 \le Z \le 1)$
        \begin{align*}
          P(0 \le Z \le 1) &= \Phi(1) - \Phi(0) \\
          &\approx .8413 - .5000 \\
          &\approx .3413
        \end{align*}
      \item[c.]
        $P(-2.50 \le Z \le 0)$
        \begin{align*}
          P(-2.50 \le Z \le 0) &= \Phi(0) - \Phi(-2.50) \\
          &\approx .5000 - .0062 \\
          &\approx .4938
        \end{align*}
      \item[d.]
        $P(-2.50 \le Z \le 2.50)$
        \begin{align*}
          P(-2.50 \le Z \le 2.50) &= \Phi(2.50) - \Phi(-2.50) \\
          &\approx .9938 - .0062 \\
          &\approx .9876
        \end{align*}
      \item[e.]
        $P(Z \le 1.37)$
        \begin{align*}
          P(Z \le 1.37) = \Phi(1.37) \approx .9147
        \end{align*}
      \item[f.]
        $P(-1.75 \le Z)$
        \begin{align*}
          P(-1.75 \le Z) &= 1 - P(Z < -1.75) \\
          &= 1 - \Phi(-1.75) \\
          &\approx 1 - .0401 \\
          &\approx .9599
        \end{align*}
      \item[g.]
        $P(-1.50 \le Z \le 2.00)$
        \begin{align*}
          P(-1.50 \le Z \le 2.00) &= \Phi(2.00) - \Phi(-1.50) \\
          &\approx .9772 - .0668 \\
          &\approx .9104
        \end{align*}
      \item[h.]
        $P(1.37 \le Z \le 2.50)$
        \begin{align*}
          P(1.37 \le Z \le 2.50) &= \Phi(2.50) - \Phi(1.37) \\
          &\approx .9938 - .9147 \\
          &\approx .0791
        \end{align*}
      \item[i.]
        $P(1.50 \le Z)$
        \begin{align*}
          P(1.50 \le Z) &= 1 - P(Z < 1.50) \\
          &= 1 - \Phi(1.50) \\
          &= 1 - .9332 \\
          &= .0668
        \end{align*}
      \item[j.]
        $P(|Z| \le 2.50)$
        \begin{align*}
          P(|Z| \le 2.50) &= P(-2.50 \le Z \le 2.50) \\
          &= \Phi(2.50) - \Phi(-2.50) \\
          &\approx .9938 - .0062 \\
          &\approx .9876
        \end{align*}
    \end{enumerate}
  \item[29.]
    In each case, determine the value of the constant $c$ that makes the probability statement correct.
    \begin{enumerate}
      \item[a.]
        $F(c) = .9838$
      \item[b.]
        $P(0 \le Z \le c) = .291$
      \item[c.]
        $P(c \le Z) = .121$
      \item[d.]
        $P(-c \le Z \le c) = .668$
      \item[e.]
        $P(c \le |Z|) = .016$
    \end{enumerate}
  \item[30.]
    Find the following percentiles for the standard normal distribution. Interpolate where appropriate.
    \begin{enumerate}
      \item[a.]
        91st
      \item[b.]
        9th
      \item[c.]
        75th
      \item[d.]
        25th
      \item[e.]
        6th
    \end{enumerate}
  \item[31.]
    Determine za for the following values of $\alpha$:
    \begin{enumerate}
      \item[a.]
        $\alpha = .0055$
      \item[b.]
        $\alpha = .09$
      \item[c.]
        $\alpha = .663$
    \end{enumerate}
  \item[32.]
    Suppose the force acting on a column that helps to support a building is a normally distributed random variable $X$ with mean value 15.0 kips and standard deviation 1.25 kips. Compute the following probabilities by standardizing and then using Table A.3.
    \begin{enumerate}
      \item[a.]
        $P(X \le 15)$
      \item[b.]
        $P(X \le 17.5)$
      \item[c.]
        $P(X \ge 10)$
      \item[d.]
        $P(14 \le X \le 18)$
      \item[e.]
        $P(|X - 15| \le 3)$
    \end{enumerate}
  \item[36.]
    Spray drift is a constant concern for pesticide applicators and agricultural producers. The inverse relationship between droplet size and drift potential is well known. The paper ``Effects of 2,4-D Formulation and Quinclorac on Spray Droplet Size and Deposition'' (\textit{Weed Technology}, 2005: 1030–1036) investigated the effects of herbicide formulation on spray atomization. A figure in the paper suggested the normal distribution with mean \qty{1050}{\micro\meter} and standard deviation \qty{150}{\micro\meter} was a reasonable model for droplet size for water (the ``control treatment'') sprayed through a \qty{760}{\milli\liter\per\minute} nozzle.
    \begin{enumerate}
      \item[a.]
        What is the probability that the size of a single droplet is less than \qty{1500}{\micro\meter}? At least \qty{1000}{\micro\meter}?
      \item[b.]
        What is the probability that the size of a single droplet is between 1000 and \qty{1500}{\micro\meter}?
      \item[c.]
        How would you characterize the smallest \qty{2}{\percent} of all droplets?
      \item[d.]
        If the sizes of five independently selected droplets are measured, what is the probability that exactly two of them exceed \qty{1500}{\micro\meter}?
    \end{enumerate}
\end{enumerate}

\end{document}
