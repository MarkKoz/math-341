\documentclass[letterpaper,12pt]{article}
\usepackage[section=3.6]{mathhw}
\usepackage{plotting}

\newcommand{\poisson}[2]{%
  \frac{e^{-#2} \cdot #2^{#1}}{#1!}%
}

\newcommand{\poissonsum}[3]{%
  \sum_{x = #1}^{#2} \poisson{x}{#3}%
}

\begin{document}

\maketitle

\begin{enumerate}
  \item[79.]
    The article ``Expectation Analysis of the Probability of Failure for Water Supply Pipes'' (\textit{J. of Pipeline Systems Engr. and Practice}, May 2012: 36–46) proposed using the Poisson distribution to model the number of failures in pipelines of various types. Suppose that for cast-iron pipe of a particular length, the expected number of failures is 1 (very close to one of the cases considered in the article). Then $X$, the number of failures, has a Poisson distribution with $\mu = 1$.
    \begin{enumerate}
      \item[a.]
        Obtain $P(X \le 5)$ by using Appendix Table A.2.
        \begin{align*}
          P(X \le 5) &= F(5; 1) = \poissonsum{0}{5}{1} \approx .999
          % &\approx \begin{aligned}[t]
          %   &.368 + (.736 - .368) + (.920 - .736) \\
          %   + &(.981 - .920) + (.996 - .981) + (.999 - .996)
          % \end{aligned} \\
          % &\approx .368 + .368 + .184 + .061 + .015 + .003 \\
        \end{align*}
      \item[b.]
        Determine $P(X = 2)$ first from the pmf formula and then from Appendix Table A.2.
        \begin{align*}
          P(X = 2) &= p(2; 1) = \poisson{2}{1} = \frac{1}{2e} \approx .184
        \end{align*}
        According to Appendix Table A.2,
        \begin{align*}
          p(2; 1) &\approx (.920 - .736) \approx .184
        \end{align*}
      \item[c.]
        Determine $P(2 \le X \le 4)$.
        \begin{align*}
          P(2 \le X \le 4) &= \poissonsum{2}{4}{1} \\
          &= \frac{1}{2!e} + \frac{1}{3!e} + \frac{1}{4!e} \\
          &\approx .184 + .061 + .015 \\
          &\approx .260
        \end{align*}
      \item[d.]
        What is the probability that $X$ exceeds its mean value by more than one standard deviation?
        \begin{align*}
          \mu &= \sigma^2 = \sigma = 1 \\
          P(X > \mu + \sigma) &= P(X > 1 + 1) \\
          &= P(X > 2) \\
          &= 1 - P(X \le 2) \\
          &= 1 - \poissonsum{0}{2}{1} \\
          &= 1 - \frac{1}{e} - \frac{1}{e} - \frac{1}{2!e} \\
          &\approx 1 - .368 - .368 - .184 \\
          &\approx 1 - .920 \\
          &\approx .080
        \end{align*}
    \end{enumerate}
  \item[80.]
    Let $X$ be the number of material anomalies occurring in a particular region of an aircraft gas-turbine disk. The article ``Methodology for Probabilistic Life Prediction of Multiple-Anomaly Materials'' (\textit{Amer. Inst. of Aeronautics and Astronautics J.}, 2006: 787–793) proposes a Poisson distribution for $X$. Suppose that $\mu = 4$.
    \begin{enumerate}
      \item[a.]
        Compute both $P(X \le 4)$ and $P(X < 4)$.
        \begin{align*}
          P(X = 4) &= \poisson{4}{4} \approx .195 \\
          \\
          P(X < 4) &= \poissonsum{0}{3}{4} \\
          &= \frac{1}{e^4} + \frac{4}{e^4} + \frac{4^2}{2!e^4} + \frac{4^3}{3!e^4} \\
          &\approx .018 + .073 + .147 + .195 \\
          &\approx .433 \\
          \\
          P(X \le 4) &= P(X < 4) + P(X = 4) \approx .433 + .195 \approx .628
        \end{align*}
      \item[b.]
        Compute $P(4 \le X \le 8)$.
        \begin{align*}
          P(4 \le X \le 8) &= \poissonsum{4}{8}{4} \\
          &= \frac{4^4}{4!e^4} + \frac{4^5}{5!e^4} + \frac{4^6}{6!e^4} + \frac{4^7}{7!e^4} + \frac{4^8}{8!e^4} \\
          &\approx .195 + .156 + .104 + .060 + .030 \\
          &\approx .545
        \end{align*}
      \item[c.]
        Compute $P(8 \le X)$.
        \begin{align*}
          P(8 \le X) &= 1 - P(X < 8) \\
          &= 1 - [P(X < 4) + P(4 \le X \le 8) - P(X = 8)] \\
          &\approx 1 - .433  - .545 + .030 \\
          &\approx .052
        \end{align*}
      \item[d.]
        What is the probability that the number of anomalies exceeds its mean value by no more than one standard deviation?
        \begin{align*}
          \mu &= 4 \\
          \sigma &= \sqrt{\sigma^2} = \sqrt{\mu} = \sqrt{4} = 2 \\
          \\
          P(X > \mu + \sigma) &= P(X > 4 + 2) \\
          &= P(X \le 6) \\
          &= P(X \le 4) + P(X = 5) + P(X = 6) \\
          &\approx .628 + \frac{4^5}{5!e^4} + \frac{4^6}{6!e^4} \\
          &\approx .628 + .156 + .104 \\
          &\approx .888
        \end{align*}
    \end{enumerate}
  \item[82.]
    Consider writing onto a computer disk and then sending it through a certifier that counts the number of missing pulses. Suppose this number $X$ has a Poisson distribution with parameter $\mu = .2$. (Suggested in ``Average Sample Number for Semi-Curtailed Sampling Using the Poisson Distribution,'' \textit{J. Quality Technology}, 1983: 126–129.)
    \begin{enumerate}
      \item[a.]
        What is the probability that a disk has exactly one missing pulse?
        \begin{align*}
          P(X = 1) &= \poisson{1}{.2} \approx .164
        \end{align*}
      \item[b.]
        What is the probability that a disk has at least two missing pulses?
        \begin{align*}
          P(X \ge 2) &= 1 - P(X < 2) \\
          &= 1 - P(X = 0) - P(X = 1) \\
          &\approx 1 - \poisson{0}{.2} - .164 \\
          &\approx 1 - .819 - .164 \\
          &\approx .017
        \end{align*}
      \item[c.]
        If two disks are independently selected, what is the probability that neither contains a missing pulse?
        \begin{align*}
          P(X = 0)^2 &\approx .819^2 \approx .671
        \end{align*}
    \end{enumerate}
  \item[84.]
    The Centers for Disease Control and Prevention reported in 2012 that 1 in 88 American children had been diagnosed with an autism spectrum disorder (ASD).
    \begin{enumerate}
      \item[a.]
        If a random sample of 200 American children is selected, what are the expected value and standard deviation of the number who have been diagnosed with ASD?
        \begin{align*}
          \mu &= np = 200 \times \frac{1}{88} \approx 2.273
        \end{align*}
        Since $n = 200 > 50$ and $np \approx 2.273 < 5$, the binomial distribution can be approximated using the Poisson distribution. Thus,
        \begin{align*}
          \sigma &= \sqrt{\sigma^2} = \sqrt{\mu} \approx \sqrt{2.272} \approx 1.508
        \end{align*}
      \item[b.]
        Referring back to (a), calculate the approximate probability that at least 2 children in the sample have been diagnosed with ASD?
        \begin{align*}
          P(X \ge 2) &= 1 - P(x < 2) \\
          &\approx 1 - \poissonsum{0}{1}{2.273} \\
          &\approx 1 - \frac{1}{e^{2.273}} - \frac{2.273}{e^{2.273}} \\
          &\approx 1 - .103 - .234 \\
          &\approx .663
        \end{align*}
      \item[c.]
        If the sample size is 352, what is the approximate probability that fewer than 5 of the selected children have been diagnosed with ASD?
        \begin{align*}
          \mu &= np = 352 \times \frac{1}{88} \approx 4
        \end{align*}
        Since $np = 4 < 5$, the Poisson distribution can still be used as an approximation.
        \begin{align*}
          P(X < 5) &= \poissonsum{0}{4}{4} \\
          &= \frac{1}{e^4} + \frac{4}{e^4} + \frac{4^2}{2!e^4} + \frac{4^3}{3!e^4} + \frac{4^4}{4!e^4} \\
          &\approx .018 + .073 + .147 + .195 + .195 \\
          &\approx .628
        \end{align*}
    \end{enumerate}
  \item[88.]
    In proof testing of circuit boards, the probability that any particular diode will fail is .01. Suppose a circuit board contains 200 diodes.
    \begin{enumerate}
      \item[a.]
        How many diodes would you expect to fail, and what is the standard deviation of the number that are expected to fail?
      \item[b.]
        What is the (approximate) probability that at least four diodes will fail on a randomly selected board?
      \item[c.]
        If five boards are shipped to a particular customer, how likely is it that at least four of them will work properly? (A board works properly only if all its diodes work.)
    \end{enumerate}
\end{enumerate}

\end{document}
