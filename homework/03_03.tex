\documentclass[letterpaper,12pt]{article}
\usepackage[section=3.3]{mathhw}
\usepackage{tabularray}

% Better alignment of \in with uppercase letters.
% https://tex.stackexchange.com/a/326323
\makeatletter
\newcommand*{\inn}{\mathrel{\mathpalette\@inn\relax}}
\def\@inn#1#2{{\setbox0=\hbox{\m@th$#1\in$}\raise\dp0\box0}}
\makeatother

\begin{document}

\maketitle

\begin{enumerate}
  \item[29.]
    The pmf of the amount of memory $X$ (GB) in a purchased flash drive was given in Example 3.13 as
    \begin{center}
      \begin{tblr}{colspec = >{$}l<{$}|*{5}c}
        x & 1 & 2 & 4 & 8 & 16 \\
        \hline
        p(x) & .05 & .10 & .35 & .40 & .10
      \end{tblr}
    \end{center}
    Compute the following:
    \begin{enumerate}
      \item[a.]
        $E(X)$
        \begin{align*}
          E(X) &= \sum_{x \inn D} x \cdot p(x) \\
          &= (1 \times .05) + (2 \times .10) + (4 \times .35) + (8 \times .40) + (16 \times .10) \\
          &= 6.45
        \end{align*}
      \item[b.]
        $V(X)$ directly from the definition
        \begin{align*}
          V(X) &= \sum_{x \inn D} (x - 6.45)^2 \cdot p(x) \\
          &= \begin{aligned}
            &.05(1 - 6.45)^2 + .10(2 - 6.45)^2 + .35(4 - 6.45)^2\\
            + &.40(8 - 6.45)^2 + .10(16 - 6.45)^2
          \end{aligned} \\
          &= 15.6475
        \end{align*}
      \item[c.]
        The standard deviation of $X$
        \begin{align*}
          \sigma = \sqrt{V(X)} = \sqrt{15.6475} \approx 3.95569
        \end{align*}
      \item[a.]
        $V(X)$ using the shortcut formula
        \begin{align*}
          V(X) &= E(X^2) - [E(X)]^2 \\
          &= \left[\sum_{x \inn D} x^2 \cdot p(x)\right] - 6.45^2 \\
          &= \begin{aligned}
            &(1^2 \times .05) + (2^2 \times .10) + (4^2 \times .35) \\
            + &(8^2 \times .40) + (16^2 \times .10) - 41.6025
          \end{aligned} \\
          &= 15.6475
        \end{align*}
    \end{enumerate}
  \item[30.]
    An individual who has automobile insurance from a certain company is randomly selected. Let $Y$ be the number of moving violations for which the individual was cited during the last 3 years. The pmf of $Y$ is
    \begin{center}
      \begin{tblr}{colspec = >{$}l<{$}|*{4}c}
        y & 0 & 1 & 2 & 3 \\
        \hline
        p(y) & .60 & .25 & .10 & .05
      \end{tblr}
    \end{center}
    \begin{enumerate}
      \item[a.]
        Compute $E(Y)$.
        \begin{align*}
          E(Y) &= \sum_{y \inn D} y \cdot p(y) \\
          &= (0 \times .60) + (1 \times .25) + (2 \times .10) + (3 \times .05) \\
          &= .6
        \end{align*}
      \item[b.]
        Suppose an individual with $Y$ violations incurs a surcharge of \$$100Y^2$. Calculate the expected amount of the surcharge.
        \begin{align*}
          E(100Y^2) &= 100 \cdot E(Y^2) \\
          &= 100 \sum_{y \inn D} y^2 \cdot p(y) \\
          &= 100 [(0^2 \times .60) + (1^2 \times .25) + (2^2 \times .10) + (3^2 \times .05)] \\
          &= \$110
        \end{align*}
    \end{enumerate}
  \item[32.]
    A certain brand of upright freezer is available in three different rated capacities: 16 ft\textsuperscript{3}, 18 ft\textsuperscript{3}, and 20 ft\textsuperscript{3}. Let $X =$ the rated capacity of a freezer of this brand sold at a certain store. Suppose that $X$ has pmf
    \begin{center}
      \begin{tblr}{colspec = >{$}l<{$}|*{3}c}
        x & 16 & 18 & 20 \\
        \hline
        p(x) & .2 & .5 & .3
      \end{tblr}
    \end{center}
    \begin{enumerate}
      \item[a.]
        Compute $E(X)$, $E(X^2)$, and $V(X)$.
        \begin{align*}
          E(X) &= \sum_{x \inn D} x \cdot p(x) \\
          &= (16 \times .2) + (18 \times .5) + (20 \times .3) \\
          &= 18.2
        \end{align*}
        \begin{align*}
          E(X^2) &= \sum_{x \inn D} x \cdot p(x) \\
          &= (16^2 \times .2) + (18^2 \times .5) + (20^2 \times .3) \\
          &= 333.2
        \end{align*}
        \begin{align*}
          V(X) = E(X^2) - [E(X)]^2 = 333.2 - 18.2^2 = 1.96
        \end{align*}
      \item[b.]
        If the price of a freezer having capacity $X$ is $70X - 650$, what is the expected price paid by the next customer to buy a freezer?
        \begin{align*}
          E(70X - 650) = 70E(X) - 650 = 70 \times 18.2 - 650 = 624
        \end{align*}
      \item[c.]
        What is the variance of the price paid by the next customer?
        \begin{align*}
          V(70X - 650) = 70^2 \cdot V(X) = 4900 \times 1.96 = 9604
        \end{align*}
      \item[d.]
        Suppose that although the rated capacity of a freezer is $X$, the actual capacity is $h(X) = X - .008X^2$. What is the expected actual capacity of the freezer purchased by the next customer?
        \begin{align*}
          E(h(x)) &= \sum_{x \inn D} (x - .008x^2) \cdot p(x) \\
          &= \sum_{x \inn D} x \cdot p(x) - .008x^2 \cdot p(x) \\
          &= E(X) - .008E(X^2) \\
          &= 18.2 - .008 \times 333.2 \\
          &= 15.5344
        \end{align*}
    \end{enumerate}
  \item[34.]
    Suppose that the number of plants of a particular type found in a rectangular sampling region (called a quadrat by ecologists) in a certain geographic area is an rv $X$ with pmf
    \begin{align*}
      p(x) = \begin{cases}
        c/x^3 & x = 1, 2, 3, \ldots \\
        0     & \text{otherwise}
      \end{cases}
    \end{align*}
    Is $E(X)$ finite? Justify your answer (this is another distribution that statisticians would call heavy-tailed).
    \begin{align*}
      E(X) &= \sum_{x = 1}^\infty x \times \frac{c}{x^3} \\
      &= c \sum_{x = 1}^\infty \frac{1}{x^2}
    \end{align*}
    This sum is a p-series where $p = 2$. According to the p-series convergence test, a p-series converges for $p > 1$. Because $p = 2 > 1$, the series converges and thus is finite. Multiplying by the constant $c$ will not change this fact.
  \item[39.]
    A chemical supply company currently has in stock 100 lb of a certain chemical, which it sells to customers in 5-lb batches. Let $X =$ the number of batches ordered by a randomly chosen customer, and suppose that $X$ has pmf
    \begin{center}
      \begin{tblr}{colspec = >{$}l<{$}|*{4}c}
        x & 1 & 2 & 3 & 4 \\
        \hline
        p(x) & .2 & .4 & .3 & .1
      \end{tblr}
    \end{center}
    Compute $E(X)$ and $V(X)$. Then compute the expected number of pounds left after the next customer’s order is shipped and the variance of the number of pounds left. [\textit{Hint:} The number of pounds left is a linear function of $X$.]
\end{enumerate}

\end{document}
