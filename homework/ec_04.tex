\documentclass[letterpaper,12pt]{article}
\usepackage[section=]{mathhw}
\usepackage{parskip}
\usepackage{amsthm}

\title{MATH 341 - Extra Point Assignment 4}

\theoremstyle{remark}
\newtheorem*{claim}{Claim}

\begin{document}

\maketitle

\begin{claim}
  If $A$ and $B$ are independent, then so are $A^c$ and $B^c$.
\end{claim}

\begin{proof}
  By De Morgan's laws,
  \begin{align*}
    P(A^c \cap B^c) = P((A \cup B)^c)
  \end{align*}
  Using the proposition $P(X) + P(X^c) = 1$,
  \begin{align*}
    P((A \cup B)^c) = 1 - P(A \cup B)
  \end{align*}
  Using the proposition $P(X \cup Y) = P(X) + P(Y) - P(X \cap Y)$,
  \begin{align*}
    1 - P(A \cup B) &= 1 - [P(A) + P(B) - P(A \cap B)] \\
    &= 1 - P(A) - P(B) + P(A \cap B)
  \end{align*}
  Using the proposition 2.8 that $X$ and $Y$ are independent iff $P(X \cap Y) = P(X) \cdot P(Y)$,
  \begin{align*}
    1 - P(A) - P(B) + P(A \cap B) &= 1 - P(A) - P(B) + P(A) \cdot P(B) \\
    &= (1 - P(A))(1 - P(B))
  \end{align*}
  Using the proposition $P(X) + P(X^c) = 1$,
  \begin{align*}
    (1 - P(A))(1 - P(B)) = P(A^c) \cdot P(B^c)
  \end{align*}
  It has been shown $P(A^c \cap B^c) = P(A^c) \cdot P(B^c)$. Thus, by proposition 2.8, $A^c$ and $B^c$ are also independent.
\end{proof}

\end{document}
